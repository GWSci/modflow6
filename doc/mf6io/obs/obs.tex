For consistency with earlier versions of MODFLOW (specifically, MODFLOW-2000 and MODFLOW-2005), \programname{} supports an ``Observation'' utility. Unlike the earlier versions of MODFLOW, the Observation utility of \programname{} does not require input of ``observed'' values, which typically were field- or lab-measured values. The Observation utility described here provides options for extracting numeric values of interest generated in the course of a model run. The Observation utility does not calculate residual values (differences between observed and model-calculated values). Output generated by the Observation utility is designed to facilitate further processing. For convenience and for consistency with earlier terminology, individual entries of the Observation utility are referred to as ``observations.''

Input for the Observation utility is read from one or more input files, where each file is associated with a specific model or package. For extracting values simulated by a GWF model, input is read from a file that is specified as type ``OBS6'' in the Name File. For extracting model values associated with a package, input is read from a file designated by the keyword ``OBS6'' in the Options block of the package of interest. The structures of observation input files for models and packages do not differ. Where a file name (or path name) containing spaces is to be read, enclose the name in single quotation marks.

Each OBS6 file can contain an OPTIONS block and one or more CONTINUOUS blocks. Each OBS6 file must contain at least one block. If present, the OPTIONS block must appear first. The CONTINUOUS blocks can be listed in any order. Comments, indicated by the presence of the ``\#'' character in column 1, can appear anywhere in the file and are ignored. 

Observations are output at the end of each time step and represent the value used by \mf during the time step. When input to the OBS utility references a stress-package boundary (for packages other than the advanced stress packages) that is not defined for a stress period of interest, a special NODATA value, indicating that a simulated value is not available, is written to output. The NODATA value is $3.0 \times 10\textsuperscript{30}$. 

Output files to be generated by the Observation utility can be either text or binary. When a text file is used for output, the user can specify the number of digits of precision are to be used in writing values. For compatibility with common spreadsheet programs, text files are written in Comma-Separated Values (CSV) format. For this reason, text output files are commonly named with ``csv'' as the extension. By convention, binary output files are named with ``bsv'' (for ``binary simulated values'') as the extension.

%When a binary file is used, the user can specify whether floating-point numbers should be written in single or double precision.

%For CONTINUOUS observations, note that boundaries identified by ID (and ID2 where used) must be defined in the corresponding package input file in all stress periods of the simulation. This requirement may mean that in some PERIOD blocks, the user will need to include entries that have no affect on the model; for example one could include a well with a recharge rate of zero or a drain boundary with a conductance of zero. In some situations preparation of input can be simplified by splitting package input into multiple input files, so that boundaries included in CONTINUOUS observations are separated from other boundaries simulated by the same package type.

\subsection{Structure of Blocks}
\vspace{5mm}

\noindent \textit{FOR EACH SIMULATION}
\lstinputlisting[style=blockdefinition]{./mf6ivar/tex/utl-obs-options.dat}
\lstinputlisting[style=blockdefinition]{./mf6ivar/tex/utl-obs-continuous.dat}

\subsection{Explanation of Variables}
\begin{description}
% DO NOT MODIFY THIS FILE DIRECTLY.  IT IS CREATED BY mf6ivar.py 

\item \textbf{Block: OPTIONS}

\begin{description}
\item \texttt{digits}---Keyword and an integer digits specifier used for conversion of simulated values to text on output. The default is 5 digits. When simulated values are written to a file specified as file type DATA in the Name File, the digits specifier controls the number of significant digits with which simulated values are written to the output file. The digits specifier has no effect on the number of significant digits with which the simulation time is written for continuous observations.

\item \texttt{PRINT\_INPUT}---keyword to indicate that the list of observation information will be written to the listing file immediately after it is read.

\end{description}
\item \textbf{Block: CONTINUOUS}

\begin{description}
\item \texttt{FILEOUT}---keyword to specify that an output filename is expected next.

\item \texttt{obs\_output\_file\_name}---Name of a file to which simulated values corresponding to observations in the block are to be written. The file name can be an absolute or relative path name. A unique output file must be specified for each SINGLE or CONTINUOUS block. If the ``BINARY'' option is used, output is written in binary form. By convention, text output files have the extension ``csv'' (for ``Comma-Separated Values'') and binary output files have the extension ``bsv'' (for ``Binary Simulated Values'').

\item \texttt{BINARY}---an optional keyword used to indicate that the output file should be written in binary (unformatted) form.

\item \texttt{obsname}---string of 1 to 40 nonblank characters used to identify the observation. The identifier need not be unique; however, identification and post-processing of observations in the output files are facilitated if each observation is given a unique name.

\item \texttt{obstype}---a string of characters used to identify the observation type.

\item \texttt{id}---Text identifying cell where observation is located. For packages other than NPF, if boundary names are defined in the corresponding package input file, ID can be a boundary name. Otherwise ID is a cellid. If the model discretization is type DIS, cellid is three integers (layer, row, column). If the discretization is DISV, cellid is two integers (layer, cell number). If the discretization is DISU, cellid is one integer (node number).

\item \texttt{id2}---Text identifying cell adjacent to cell identified by ID. The form of ID2 is as described for ID. ID2 is used for intercell-flow observations of a GWF model, for three observation types of the LAK Package, for two observation types of the MAW Package, and one observation type of the UZF Package.

\end{description}


\end{description}


\subsection{Available Observation Types}

\subsubsection{GWF Observations}
Observations are available for GWF models, GWF-GWF exchanges, and all stress packages. Available observation types have been listed for each package that supports observations (tables~\ref{table:gwfobstype} to~\ref{table:gwf-gwfobstype}). All available observation types are repeated in Table~\ref{table:gwf-obstypetable} for convenience. 

The sign convention adopted for flow observations are identical to the conventions used in budgets contained in listing files and used in the cell-by-cell budget output. For flow-ja-face observation types, negative and positive values represent a loss from and gain to the cellid specified for ID, respectively. For standard stress packages (Package = CHD, DRN, EVT, GHB, RCH, RIV, and WEL), negative and positive values represent a loss from and gain to the GWF model, respectively. For advanced packages (Package = LAK, MAW, SFR, and UZF), negative and positive values for exchanges with the GWF model (Observation type = lak, maw, sfr, uzf-gwrch, uzf-gwd, uzf-gwd-to-mvr, and uzf-gwet) represent a loss from and gain to the GWF model, respectively. For other advanced stress package flow terms, negative and positive values represent a loss from and gain from the advanced package, respectively.

\FloatBarrier

\begingroup
\makeatletter
\ifx\LT@ii\@undefined\else
\def\LT@entry#1#2{\noexpand\LT@entry{-#1}{#2}}
\xdef\LT@i{\LT@ii}
\fi
\endgroup

% model observations
\begin{longtable}{p{2cm} p{2.75cm} p{2cm} p{1.25cm} p{7cm}}
\caption{Available observation types for the GWF Model} \tabularnewline

\hline
\hline
\textbf{Model} & \textbf{Observation types} & \textbf{ID} & \textbf{ID2} & \textbf{Description} \\
\hline
\endfirsthead

\captionsetup{textformat=simple}
\caption*{\textbf{Table \arabic{table}.}{\quad}List of symbols used in this report.---Continued} \\

\hline
\hline
\textbf{Model} & \textbf{Observation types} & \textbf{ID} & \textbf{ID2} & \textbf{Description} \\
\hline
\endhead

\hline
\endfoot


GWF Model observations include the simulated groundwater head (\texttt{head}), calculated drawdown (\texttt{drawdown}) at a node, and the flow between two connected nodes (\texttt{flow-ja-face}). The data required for each GWF Model observation type is defined in table~\ref{table:gwfobstype}. For \texttt{flow-ja-face} observation types, negative and positive values represent a loss from and gain to the \texttt{cellid} specified for ID, respectively.

\subsubsection{Structure of Blocks}
\vspace{5mm}

\noindent \textit{FOR EACH SIMULATION}
\lstinputlisting[style=blockdefinition]{./mf6ivar/tex/utl-obs-options.dat}
\lstinputlisting[style=blockdefinition]{./mf6ivar/tex/utl-obs-continuous.dat}

\subsubsection{Explanation of Variables}
\begin{description}
\input{./mf6ivar/tex/utl-obs-desc.tex}
\end{description}


\begin{longtable}{p{2cm} p{2.75cm} p{2cm} p{1.25cm} p{7cm}}
\caption{Available GWF model observation types} \tabularnewline

\hline
\hline
\textbf{Model} & \textbf{Observation type} & \textbf{ID} & \textbf{ID2} & \textbf{Description} \\
\hline
\endhead

\hline
\endfoot

\input{../Common/gwf-obs.tex}
\label{table:gwfobstype}
\end{longtable}

\vspace{5mm}
\subsubsection{Example Observation Input File}

An example GWF Model observation file is shown below.

\lstinputlisting[style=inputfile]{./mf6ivar/examples/utl-obs-example-obs.dat}


\end{longtable}
\addtocounter{table}{-1}

% internal packages
\begin{longtable}{p{2cm} p{2.75cm} p{2cm} p{1.25cm} p{7cm}}
\hline
\hline
\textbf{Internal Package} & \textbf{Observation type} & \textbf{ID} & \textbf{ID2} & \textbf{Description} \\
\hline
\endfirsthead

\captionsetup{textformat=simple}
\caption*{\textbf{Table \arabic{table}.}{\quad}Available observation types.---Continued} \\

\hline
\hline
\textbf{Internal Package} & \textbf{Observation types} & \textbf{ID} & \textbf{ID2} & \textbf{Description} \\
\hline
\endhead

\hline
\endfoot

CSUB & csub & icsubno or boundname & -- & Flow between the groundwater system and a interbed or group of interbeds. \\
CSUB & inelastic-csub & icsubno or boundname & -- & Flow between the groundwater system and a interbed or group of interbeds from inelastic compaction. \\
CSUB & elastic-csub & icsubno or boundname & -- & Flow between the groundwater system and a interbed or group of interbeds from elastic compaction. \\
CSUB & coarse-csub & cellid & -- & Flow between the groundwater system and coarse-grained materials in a GWF cell. \\
CSUB & csub-cell & cellid & -- & Flow between the groundwater system for all interbeds and coarse-grained materials in a GWF cell. \\
CSUB & wcomp-csub-cell & cellid & -- & Flow between the groundwater system for all interbeds and coarse-grained materials in a GWF cell from water compressibility. \\

CSUB & sk & icsubno or boundname & -- & Convertible interbed storativity in a interbed or group of interbeds. Convertible interbed storativity is inelastic interbed storativity if the current effective stress is greater than the preconsolidation stress. \\
CSUB & ske & icsubno or boundname & -- & Elastic interbed storativity in a interbed or group of interbeds. \\
CSUB & sk-cell & cellid & -- & Convertible interbed and coarse-grained material storativity in a GWF cell. Convertible interbed storativity is inelastic interbed storativity if the current effective stress is greater than the preconsolidation stress. \\
CSUB & ske-cell & cellid & -- & Elastic interbed and coarse-grained material storativity in a GWF cell. \\

CSUB & estress-cell & cellid & -- & effective stress in a GWF cell. \\
CSUB & gstress-cell & cellid & -- & geostatic stress in a GWF cell. \\

CSUB & interbed-compaction & icsubno or boundname  & -- & interbed compaction in a interbed or group of interbeds. \\
CSUB & inelastic-compaction &  icsubno or boundname & -- & inelastic interbed compaction in a interbed or group of interbeds. \\
CSUB & elastic-compaction &  icsubno or boundname & -- & elastic interbed compaction a interbed or group of interbeds. \\
CSUB & coarse-compaction & cellid  & -- & elastic compaction in coarse-grained materials in a GWF cell. \\
CSUB & compaction-cell & cellid  & -- & total compaction in coarse-grained materials and all interbeds in a GWF cell. \\

CSUB & thickness & icsubno or boundname & -- & thickness of a interbed or group of interbeds. \\
CSUB & coarse-thickness & cellid  & -- & thickness of coarse-grained materials in a GWF cell. \\
CSUB & thickness-cell & cellid  & -- & total thickness of coarse-grained materials and all interbeds in a GWF cell. \\

CSUB & theta & icsubno & -- & porosity of a interbed . \\
CSUB & coarse-theta & cellid  & -- & porosity of coarse-grained materials in a GWF cell. \\
CSUB & theta-cell & cellid  & -- & thickness-weighted porosity of coarse-grained materials and all interbeds in a GWF cell. \\

CSUB & delay-flowtop & icsubno  & -- & Flow between the groundwater system and a delay interbed across the top of the interbed. \\
CSUB & delay-flowbot & icsubno  & -- & Flow between the groundwater system and a delay interbed across the bottom of the interbed. \\

CSUB & delay-head & icsubno  & idcellno & head in interbed delay cell idcellno (1 $<=$ idcellno $<=$ NDELAYCELLS). \\
CSUB & delay-gstress & icsubno  & idcellno & geostatic stress in interbed delay cell idcellno (1 $<=$ idcellno $<=$ NDELAYCELLS). \\
CSUB & delay-estress & icsubno  & idcellno & effective stress in interbed delay cell idcellno (1 $<=$ idcellno $<=$ NDELAYCELLS). \\
CSUB & delay-preconstress & icsubno  & idcellno & preconsolidation stress in interbed delay cell idcellno (1 $<=$ idcellno $<=$ NDELAYCELLS). \\
CSUB & delay-compaction & icsubno  & idcellno & compaction in interbed delay cell idcellno (1 $<=$ idcellno $<=$ NDELAYCELLS). \\
CSUB & delay-thickness & icsubno  & idcellno & thickness of interbed delay cell idcellno (1 $<=$ idcellno $<=$ NDELAYCELLS). \\
CSUB & delay-theta & icsubno  & idcellno & porosity of interbed delay cell idcellno (1 $<=$ idcellno $<=$ NDELAYCELLS). \\

CSUB & preconstress-cell & cellid  & -- & preconsolidation stress in a GWF cell containing at least one interbed.


\end{longtable}
\addtocounter{table}{-1}

% stress packages
\begin{longtable}{p{2cm} p{2.75cm} p{2cm} p{1.25cm} p{7cm}}
\hline
\hline
\textbf{Stress Package} & \textbf{Observation type} & \textbf{ID} & \textbf{ID2} & \textbf{Description} \\
\hline
\endfirsthead

\captionsetup{textformat=simple}
\caption*{\textbf{Table \arabic{table}.}{\quad}Available observation types.---Continued} \\

\hline
\hline
\textbf{Stress Package} & \textbf{Observation types} & \textbf{ID} & \textbf{ID2} & \textbf{Description} \\
\hline
\endhead

\hline
\endfoot

CHD & chd & cellid or boundname & -- & Flow between the groundwater system and a constant-head boundary or a group of cells with constant-head boundaries.
 \\
DRN & drn & cellid or boundname & -- & Flow between the groundwater system and a drain boundary or group of drain boundaries. \\
DRN & to-mvr & cellid or boundname & -- & Drain boundary discharge that is available for the MVR package for a drain boundary or a group of drain boundaries.
 \\
EVT & evt & cellid or boundname & -- & Flow from the groundwater system through an evapotranspiration boundary or group of evapotranspiration boundaries. \\
GHB & ghb & cellid or boundname & -- & Flow between the groundwater system and a general-head boundary or group of general-head boundaries. \\
GHB & to-mvr & cellid or boundname & -- & General-head boundary discharge that is available for the MVR package from a general-head boundary or group of general-head boundaries. \\
RCH & rch & cellid or boundname & -- & Flow to the groundwater system through a recharge boundary or a group of recharge boundaries. \\
RIV & riv & cellid or boundname & -- & Flow between the groundwater system and a river boundary. \\
RIV & to-mvr & cellid or boundname & -- & River boundary discharge that is available for the MVR package. \\
WEL & wel & cellid or boundname & -- & Flow between the groundwater system and a well boundary or a group of well boundaries. \\
WEL & to-mvr & cellid or boundname & -- & Well boundary discharge that is available for the MVR package for a well boundary or a group of well boundaries. \\
\hline
LAK & stage & lakeno or boundname & -- & Surface-water stage in a lake. If boundname is specified, boundname must be unique for each lake. \\
LAK & ext-inflow & lakeno or boundname & -- & Specified inflow into a lake or group of lakes. \\
LAK & outlet-inflow & lakeno or boundname & -- & Simulated inflow from upstream lake outlets into a lake or group of lakes. \\
LAK & inflow & lakeno or boundname & -- & Sum of specified inflow and simulated inflow from upstream lake outlets into a lake or group of lakes. \\
LAK & from-mvr & lakeno or boundname & -- & Inflow into a lake or group of lakes from the MVR package. \\
LAK & rainfall & lakeno or boundname & -- & Rainfall rate applied to a lake or group of lakes. \\
LAK & runoff & lakeno or boundname & -- & Runoff rate applied to a lake or group of lakes. \\
LAK & lak & lakeno or boundname & \texttt{iconn} or -- & Simulated flow rate for a lake or group of lakes and its aquifer connection(s). If boundname is not specified for ID, then the simulated lake-aquifer flow rate at a specific lake connection is observed. In this case, ID2 must be specified and is the connection number \texttt{iconn}. \\
LAK & withdrawal & lakeno or boundname & -- & Specified withdrawal rate from a lake or group of lakes. \\
LAK & evaporation & lakeno or boundname & -- & Simulated evaporation rate from a lake or group of lakes. \\
LAK & ext-outflow & outletno or boundname & -- & External outflow from a lake outlet, a lake, or a group of lakes to an external boundary. If boundname is not specified for ID, then the external outflow from a specific lake outlet is observed. In this case, ID is the outlet number outletno. \\
LAK & to-mvr & outletno or boundname & -- & Outflow from a lake outlet, a lake, or a group of lakes that is available for the MVR package. If boundname is not specified for ID, then the outflow available for the MVR package from a specific lake outlet is observed. In this case, ID is the outlet number outletno. \\
LAK & storage & lakeno or boundname & -- & Simulated storage flow rate for a lake or group of lakes. \\
LAK & constant & lakeno or boundname & -- & Simulated constant-flow rate for a lake or group of lakes. \\
LAK & outlet & outletno or boundname & -- & Simulated outlet flow rate from a lake outlet, a lake, or a group of lakes. If boundname is not specified for ID, then the flow from a specific lake outlet is observed. In this case, ID is the outlet number outletno. \\
LAK & volume & lakeno or boundname & -- & Simulated lake volume or group of lakes. \\
LAK & surface-area & lakeno or boundname & -- & Simulated surface area for a lake or group of lakes. \\
LAK & wetted-area & lakeno or boundname & \texttt{iconn} or -- & Simulated wetted-area for a lake or group of lakes and its aquifer connection(s). If boundname is not specified for ID, then the wetted area of a specific lake connection is observed. In this case, ID2 must be specified and is the connection number \texttt{iconn}. \\
LAK & conductance & lakeno or boundname & \texttt{iconn} or -- & Calculated conductance for a lake or group of lakes and its aquifer connection(s). If boundname is not specified for ID, then the calculated conductance of a specific lake connection is observed. In this case, ID2 must be specified and is the connection number \texttt{iconn}.
 \\
\hline
MAW & head & wellno or boundname & -- & Head in a multi-aquifer well. If boundname is specified, boundname must be unique for each multi-aquifer well. \\
MAW & from-mvr & wellno or boundname & -- & Simulated inflow to a well from the MVR package for a multi-aquifer well or a group of multi-aquifer wells. \\
MAW & maw & wellno or boundname & \texttt{icon} or -- & Simulated flow rate for a multi-aquifer well or a group of multi-aquifer wells and its aquifer connection(s). If boundname is not specified for ID, then the simulated multi-aquifer well-aquifer flow rate at a specific multi-aquifer well connection is observed. In this case, ID2 must be specified and is the connection number \texttt{icon}. \\
MAW & rate & wellno or boundname & -- & Simulated pumping rate for a multi-aquifer well or a group of multi-aquifer wells. \\
MAW & rate-to-mvr & wellno or boundname & -- & Simulated well discharge that is available for the MVR package for a multi-aquifer well or a group of multi-aquifer wells. \\
MAW & fw-rate & wellno or boundname & -- & Simulated flowing well flow rate for a multi-aquifer well or a group of multi-aquifer wells.  \\
MAW & fw-to-mvr & wellno or boundname & -- & Simulated flowing well discharge rate that is available for the MVR package for a multi-aquifer well or a group of multi-aquifer wells. \\
MAW & storage & wellno or boundname & -- & Simulated storage flow rate for a multi-aquifer well or a group of multi-aquifer wells. \\
MAW & constant & wellno or boundname & -- & Simulated constant-flow rate for a multi-aquifer well or a group of multi-aquifer wells. \\
MAW & conductance & wellno or boundname & \texttt{icon} or -- & Simulated well conductance for a multi-aquifer well or a group of multi-aquifer wells and its aquifer connection(s). If boundname is not specified for ID, then the simulated multi-aquifer well conductance at a specific multi-aquifer well connection is observed. In this case, ID2 must be specified and is the connection number \texttt{icon}. \\
MAW & fw-conductance & wellno or boundname & -- & Simulated flowing well conductance for a multi-aquifer well or a group of multi-aquifer wells. \\
\hline
SFR & stage & rno or boundname & -- & Surface-water stage in a stream-reach boundary. If boundname is specified, boundname must be unique for each reach. \\
SFR & ext-inflow & rno or boundname & -- & Inflow into a stream-reach from an external boundary for a stream-reach or a group of stream-reaches. \\
SFR & inflow & rno or boundname & -- & Inflow into a stream-reach from upstream reaches for a stream-reach or a group of stream-reaches. \\
SFR & from-mvr & rno or boundname & -- & Inflow into a stream-reach from the MVR package for a stream-reach or a group of stream-reaches. \\
SFR & rainfall & rno or boundname & -- & Rainfall rate applied to a stream-reach or a group of stream-reaches. \\
SFR & runoff & rno or boundname & -- & Runoff rate applied to a stream-reach or a group of stream-reaches. \\
SFR & sfr & rno or boundname & -- & Simulated flow rate for a stream-reach and its aquifer connection for a stream-reach or a group of stream-reaches. \\
SFR & evaporation & rno or boundname & -- & Simulated evaporation rate from a stream-reach or a group of stream-reaches. \\
SFR & outflow & rno or boundname & -- & Outflow from a stream-reach to downstream reaches for a stream-reach or a group of stream-reaches. \\
SFR & ext-outflow & rno or boundname & -- & Outflow from a stream-reach to an external boundary for a stream-reach or a group of stream-reaches. \\
SFR & to-mvr & rno or boundname & -- & Outflow from a stream-reach that is available for the MVR package for a stream-reach or a group of stream-reaches. \\
SFR & upstream-flow & rno or boundname & -- & Upstream flow for a stream-reach or a group of stream-reaches from upstream reaches and the MVR package. \\
SFR & downstream-flow & rno or boundname & -- & Downstream flow for a stream-reach or a group of stream-reaches prior to diversions and the MVR package.
 \\
\hline
UZF & uzf-gwrch & iuzno or boundname & -- & Simulated recharge to the aquifer calculated by the UZF package for a UZF cell or a group of UZF cells.\\
UZF & uzf-gwd & iuzno or boundname & -- & Simulated groundwater discharge to the land surface calculated by the UZF package for a UZF cell or a group of UZF cells. \\
UZF & uzf-gwd-to-mvr & iuzno or boundname & -- & Simulated groundwater discharge to the land surface calculated by the UZF package that is available to the MVR package for a UZF cell or a group of UZF cells. \\
UZF & uzf-gwet & iuzno or boundname & -- & Simulated groundwater evapotranspiration calculated by the UZF package for a UZF cell or a group of UZF cells.\\
UZF & infiltration & iuzno or boundname & -- & Specified infiltration rate applied to a UZF package for a UZF cell or a group of UZF cells with landflag values not equal to zero.\\
UZF & from-mvr & iuzno or boundname & -- & Inflow into a UZF cell from the MVR package for a UZF cell or a group of UZF cells. \\
UZF & rej-inf & iuzno or boundname & -- & Simulated rejected infiltration calculated by the UZF package for a UZF cell or a group of UZF cells. \\
UZF & rej-inf-to-mvr & iuzno or boundname & -- & Simulated rejected infiltration calculated by the UZF package that is available to the MVR package for a UZF cell or a group of UZF cells. \\
UZF & uzet & iuzno or boundname & -- & Simulated unsaturated evapotranspiration calculated by the UZF package for a UZF cell or a group of UZF cells.\\
UZF & storage & iuzno or boundname & -- & Simulated storage flow rate for a UZF package cell or a group of UZF cells. \\
UZF & net-infiltration & iuzno or boundname & -- & Simulated net infiltration rate for a UZF package cell or a group of UZF cells. \\
UZF & water-content & iuzno or boundname & depth & Unsaturated-zone water content at a user-specified depth (ID2) relative to the top of GWF cellid for a UZF cell. The user-specified depth must be greater than or equal to zero and less than the thickness of GWF cellid (TOP - BOT). If boundname is specified, boundname must be unique for each UZF cell.
\label{table:gwf-obstypetable}
\end{longtable}
\addtocounter{table}{-1}

% exchange
\begin{longtable}{p{2cm} p{2.75cm} p{2cm} p{1.25cm} p{7cm}}
\hline
\hline
\textbf{Exchange} & \textbf{Observation type} & \textbf{ID} & \textbf{ID2} & \textbf{Description} \\
\hline
\endfirsthead

\captionsetup{textformat=simple}
\caption*{\textbf{Table \arabic{table}.}{\quad}Available observation types.---Continued} \\

\hline
\hline
\textbf{Exchange} & \textbf{Observation types} & \textbf{ID} & \textbf{ID2} & \textbf{Description} \\
\hline
\endhead

\hline
\endfoot

GWF-GWF & flow-ja-face & cellid & cellid & Flow rate for specified exchange.
\end{longtable}

\normalsize

\FloatBarrier


\subsubsection{GWT Observations}
Observations are available for GWT models and GWT stress packages. Available observation types have been listed for each package that supports observations (tables~\ref{table:gwtobstype} to~\ref{table:gwt-uztobstype}). All available observation types are repeated in Table~\ref{table:gwt-obstypetable} for convenience. 

The sign convention adopted for transport observations are identical to the conventions used in budgets contained in listing files and used in the cell-by-cell budget output. For flow-ja-face observation types, negative and positive values represent a loss from and gain to the cellid specified for ID, respectively. For standard stress packages, negative and positive values represent a loss from and gain to the GWT model, respectively. For advanced transport packages (Package = LKT, MWT, SFT, and UZT), negative and positive values for exchanges with the GWT model (Observation type = lkt, mwt, sft, and uzt) represent a loss from and gain to the GWT model, respectively. For other advanced stress package flow terms, negative and positive values represent a loss from and gain from the advanced package, respectively.

\FloatBarrier

\begingroup
\makeatletter
\ifx\LT@ii\@undefined\else
\def\LT@entry#1#2{\noexpand\LT@entry{-#1}{#2}}
\xdef\LT@i{\LT@ii}
\fi
\endgroup

% model observations
\begin{longtable}{p{2cm} p{2.75cm} p{2cm} p{1.25cm} p{7cm}}
\caption{Available observation types for the GWT Model} \tabularnewline

\hline
\hline
\textbf{Model} & \textbf{Observation types} & \textbf{ID} & \textbf{ID2} & \textbf{Description} \\
\hline
\endfirsthead

\captionsetup{textformat=simple}
\caption*{\textbf{Table \arabic{table}.}{\quad}List of symbols used in this report.---Continued} \\

\hline
\hline
\textbf{Model} & \textbf{Observation types} & \textbf{ID} & \textbf{ID2} & \textbf{Description} \\
\hline
\endhead

\hline
\endfoot


GWT Model observations include the simulated groundwater concentration (\texttt{concentration}), and the mass flow, with units of mass per time, between two connected cells (\texttt{flow-ja-face}). The data required for each GWT Model observation type is defined in table~\ref{table:gwtobstype}. For \texttt{flow-ja-face} observation types, negative and positive values represent a loss from and gain to the \texttt{cellid} specified for ID, respectively.

\begin{longtable}{p{2cm} p{2.75cm} p{2cm} p{1.25cm} p{7cm}}
\caption{Available GWT model observation types} \tabularnewline

\hline
\hline
\textbf{Model} & \textbf{Observation type} & \textbf{ID} & \textbf{ID2} & \textbf{Description} \\
\hline
\endhead

\hline
\endfoot

\input{../Common/gwt-obs.tex}
\label{table:gwtobstype}
\end{longtable}

\vspace{5mm}
\subsubsection{Example Observation Input File}

An example GWT Model observation file is shown below.

\lstinputlisting[style=inputfile]{./mf6ivar/examples/utl-obs-gwt-example.dat}


\end{longtable}
\addtocounter{table}{-1}

% stress packages
\begin{longtable}{p{2cm} p{2.75cm} p{2cm} p{1.25cm} p{7cm}}
\hline
\hline
\textbf{Stress Package} & \textbf{Observation type} & \textbf{ID} & \textbf{ID2} & \textbf{Description} \\
\hline
\endfirsthead

\captionsetup{textformat=simple}
\caption*{\textbf{Table \arabic{table}.}{\quad}Available observation types.---Continued} \\

\hline
\hline
\textbf{Stress Package} & \textbf{Observation types} & \textbf{ID} & \textbf{ID2} & \textbf{Description} \\
\hline
\endhead

\hline
\endfoot

CNC & cnc & cellid or boundname & -- & Mass flow between the groundwater system and a constant-concentration boundary or a group of cells with constant-concentration boundaries.
 \\
\hline
SRC & src & cellid or boundname & -- & Mass source loading rate between the groundwater system and a mass source loading boundary or a group of  boundaries. \\
\hline
% general APT observations
SFT & concentration & rno or boundname & -- & Reach concentration. If boundname is specified, boundname must be unique for each reach. \\
SFT & flow-ja-face & rno or boundname & rno or -- & Mass flow between two reaches.  If a boundname is specified for ID1, then the result is the total mass flow for all reaches. If a boundname is specified for ID1 then ID2 is not used.\\
SFT & storage & rno or boundname & -- & Simulated mass storage flow rate for a reach or group of reaches. \\
SFT & constant & rno or boundname & -- & Simulated mass constant-flow rate for a reach or group of reaches. \\
SFT & from-mvr & rno or boundname & -- & Simulated mass inflow into a reach or group of reaches from the MVT package. Mass inflow is calculated as the product of provider concentration and the mover flow rate. \\
SFT & to-mvr & rno or boundname & -- & Mass outflow from a reach, or a group of reaches that is available for the MVR package. If boundname is not specified for ID, then the outflow available for the MVR package from a specific reach is observed. \\
SFT & sft & rno or boundname & \texttt{iconn} or -- & Mass flow rate for a reach or group of reaches and its aquifer connection(s). If boundname is not specified for ID, then the simulated reach-aquifer flow rate at a specific reach connection is observed. In this case, ID2 must be specified and is the connection number \texttt{iconn} for reach \texttt{rno}. \\

%observations specific to the stream transport package
% rainfall evaporation runoff ext-inflow withdrawal outflow
SFT & rainfall & rno or boundname & -- & Rainfall rate applied to a reach or group of reaches multiplied by the rainfall concentration. \\
SFT & evaporation & rno or boundname & -- & Simulated evaporation rate from a reach or group of reaches multiplied by the evaporation concentration. \\
SFT & runoff & rno or boundname & -- & Runoff rate applied to a reach or group of reaches multiplied by the runoff concentration. \\
SFT & ext-inflow & rno or boundname & -- & Mass inflow into a reach or group of reaches calculated as the external inflow rate multiplied by the inflow concentration. \\
SFT & ext-outflow & rno or boundname & -- & External outflow from a reach or group of reaches to an external boundary. If boundname is not specified for ID, then the external outflow from a specific reach is observed. In this case, ID is the reach rno.
 \\
\hline
% general APT observations
LKT & concentration & lakeno or boundname & -- & Lake concentration. If boundname is specified, boundname must be unique for each lake. \\
LKT & flow-ja-face & lakeno or boundname & lakeno or -- & Mass flow between two lakes connected by an outlet.  If more than one outlet is used to connect the same two lakes, then the mass flow for only the first outlet can be observed.  If a boundname is specified for ID1, then the result is the total mass flow for all outlets for a lake. If a boundname is specified for ID1 then ID2 is not used.\\
LKT & storage & lakeno or boundname & -- & Simulated mass storage flow rate for a lake or group of lakes. \\
LKT & constant & lakeno or boundname & -- & Simulated mass constant-flow rate for a lake or group of lakes. \\
LKT & from-mvr & lakeno or boundname & -- & Simulated mass inflow into a lake or group of lakes from the MVT package. Mass inflow is calculated as the product of provider concentration and the mover flow rate. \\
LKT & to-mvr & outletno or boundname & -- & Mass outflow from a lake outlet, a lake, or a group of lakes that is available for the MVR package. If boundname is not specified for ID, then the outflow available for the MVR package from a specific lake outlet is observed. In this case, ID is the outlet number, which must be between 1 and NOUTLETS. \\
LKT & lkt & lakeno or boundname & \texttt{iconn} or -- & Mass flow rate for a lake or group of lakes and its aquifer connection(s). If boundname is not specified for ID, then the simulated lake-aquifer flow rate at a specific lake connection is observed. In this case, ID2 must be specified and is the connection number \texttt{iconn} for lake \texttt{lakeno}. \\

%observations specific to the lake package
% rainfall evaporation runoff ext-inflow withdrawal outflow
LKT & rainfall & lakeno or boundname & -- & Rainfall rate applied to a lake or group of lakes multiplied by the rainfall concentration. \\
LKT & evaporation & lakeno or boundname & -- & Simulated evaporation rate from a lake or group of lakes multiplied by the evaporation concentration. \\
LKT & runoff & lakeno or boundname & -- & Runoff rate applied to a lake or group of lakes multiplied by the runoff concentration. \\
LKT & ext-inflow & lakeno or boundname & -- & Mass inflow into a lake or group of lakes calculated as the external inflow rate multiplied by the inflow concentration. \\
LKT & withdrawal & lakeno or boundname & -- & Specified withdrawal rate from a lake or group of lakes multiplied by the simulated lake concentration. \\
LKT & ext-outflow & outletno or boundname & -- & External outflow from a lake outlet, a lake, or a group of lakes to an external boundary. If boundname is not specified for ID, then the external outflow from a specific lake outlet is observed. In this case, ID is the outlet number outletno.

%LKT & outlet-inflow & lakeno or boundname & -- & Simulated inflow from upstream lake outlets into a lake or group of lakes. \\
%LKT & inflow & lakeno or boundname & -- & Sum of specified inflow and simulated inflow from upstream lake outlets into a lake or group of lakes. \\
%LKT & outlet & outletno or boundname & -- & Simulate outlet flow rate from a lake outlet, a lake, or a group of lakes. If boundname is not specified for ID, then the flow from a specific lake outlet is observed. In this case, ID is the outlet number outletno. \\
%LKT & volume & lakeno or boundname & -- & Simulated lake volume or group of lakes. \\
%LKT & surface-area & lakeno or boundname & -- & Simulated surface area for a lake or group of lakes. \\
%LKT & wetted-area & lakeno or boundname & \texttt{iconn} or -- & Simulated wetted-area for a lake or group of lakes and its aquifer connection(s). If boundname is not specified for ID, then the wetted area of a specific lake connection is observed. In this case, ID2 must be specified and is the connection number \texttt{iconn}. \\
%LKT & conductance & lakeno or boundname & \texttt{iconn} or -- & Calculated conductance for a lake or group of lakes and its aquifer connection(s). If boundname is not specified for ID, then the calculated conductance of a specific lake connection is observed. In this case, ID2 must be specified and is the connection number \texttt{iconn}.
 \\
\hline
% general APT observations
MWT & concentration & mawno or boundname & -- & Well concentration. If boundname is specified, boundname must be unique for each well. \\
%flowjaface not included
MWT & storage & mawno or boundname & -- & Simulated mass storage flow rate for a well or group of wells. \\
MWT & constant & mawno or boundname & -- & Simulated mass constant-flow rate for a well or group of wells. \\
MWT & from-mvr & mawno or boundname & -- & Simulated mass inflow into a well or group of wells from the MVT package. Mass inflow is calculated as the product of provider concentration and the mover flow rate. \\
MWT & mwt & mawno or boundname & \texttt{iconn} or -- & Mass flow rate for a well or group of wells and its aquifer connection(s). If boundname is not specified for ID, then the simulated well-aquifer flow rate at a specific well connection is observed. In this case, ID2 must be specified and is the connection number \texttt{iconn} for well \texttt{mawno}. \\

% observations specific to the mwt package
MWT & rate & mawno or boundname & -- & Simulated mass flow rate for a well or group of wells. \\
MWT & fw-rate & mawno or boundname & -- & Simulated mass flow rate for a flowing well or group of flowing wells. \\
MWT & rate-to-mvr & well or boundname & -- & Simulated mass flow rate that is sent to the MVT Package for a well or group of wells.\\
MWT & fw-rate-to-mvr & well or boundname & -- & Simulated mass flow rate that is sent to the MVT Package from a flowing well or group of flowing wells. \\
 \\
\hline
% general APT observations
UZT & concentration & uztno or boundname & -- & uzt cell concentration. If boundname is specified, boundname must be unique for each uzt cell. \\
UZT & flow-ja-face & uztno or boundname & uztno or -- & Mass flow between two uzt cells.  If a boundname is specified for ID1, then the result is the total mass flow for all uzt cells. If a boundname is specified for ID1 then ID2 is not used.\\
UZT & storage & uztno or boundname & -- & Simulated mass storage flow rate for a uzt cell or group of uzt cells. \\
UZT & from-mvr & uztno or boundname & -- & Simulated mass inflow into a uzt cell or group of uzt cells from the MVT package. Mass inflow is calculated as the product of provider concentration and the mover flow rate. \\
UZT & uzt & uztno or boundname & \texttt{iconn} or -- & Mass flow rate for a uzt cell or group of uzt cells and its aquifer connection(s). If boundname is not specified for ID, then the simulated uzt-aquifer flow rate at a specific uzt cell connection is observed. In this case, ID2 must be specified and is the connection number for the uzt cell. \\

%observations specific to the uzt package
% infiltration rej-inf uzet rej-inf-to-mvr
UZT & infiltration & uztno or boundname & -- & Infiltration rate applied to a uzt cell or group of uzt cells multiplied by the infiltration concentration. \\
UZT & rej-inf & uztno or boundname & -- & Rejected infiltration rate applied to a uzt cell or group of uzt cells multiplied by the infiltration concentration. \\
UZT & uzet & uztno or boundname & -- & Unsaturated zone evapotranspiration rate applied to a uzt cell or group of uzt cells multiplied by the uzt cell concentration. \\
UZT & infiltration & uztno or boundname & -- & Rejected infiltration rate applied to a uzt cell or group of uzt cells multiplied by the infiltration concentration that is sent to the mover package. \\

\label{table:gwt-obstypetable}
\end{longtable}
%\addtocounter{table}{-1}


\normalsize

\FloatBarrier


