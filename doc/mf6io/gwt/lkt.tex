Lake Transport (LKT) Package information is read from the file that is specified by ``LKT6'' as the file type.  There can be as many LKT Packages as necessary for a GWT model. Each LKT Package is designed to work with flows from a single corresponding GWF LAK Package. By default \mf uses the LKT package name to determine which LAK Package corresponds to the LKT Package.  Therefore, the package name of the LKT Package (as specified in the GWT name file) must match with the name of the corresponding LAK Package (as specified in the GWF name file).  Alternatively, the name of the flow package can be specified using the FLOW\_PACKAGE\_NAME keyword in the options block.  The GWT LKT Package cannot be used without a corresponding GWF LAK Package.

The LKT Package does not have a dimensions block; instead, dimensions for the LKT Package are set using the dimensions from the corresponding LAK Package.  For example, the LAK Package requires specification of the number of lakes (NLAKES).  LKT sets the number of lakes equal to NLAKES.  Therefore, the PACKAGEDATA block below must have NLAKES entries in it.

\vspace{5mm}
\subsubsection{Structure of Blocks}
\lstinputlisting[style=blockdefinition]{./mf6ivar/tex/gwt-lkt-options.dat}
\lstinputlisting[style=blockdefinition]{./mf6ivar/tex/gwt-lkt-packagedata.dat}
\lstinputlisting[style=blockdefinition]{./mf6ivar/tex/gwt-lkt-period.dat}

\vspace{5mm}
\subsubsection{Explanation of Variables}
\begin{description}
% DO NOT MODIFY THIS FILE DIRECTLY.  IT IS CREATED BY mf6ivar.py 

\item \textbf{Block: OPTIONS}

\begin{description}
\item \texttt{flow\_package\_name}---keyword to specify the name of the corresponding flow package.  If not specified, then the corresponding flow package must have the same name as this advanced transport package (the name associated with this package in the GWT name file).

\item \texttt{auxiliary}---defines an array of one or more auxiliary variable names.  There is no limit on the number of auxiliary variables that can be provided on this line; however, lists of information provided in subsequent blocks must have a column of data for each auxiliary variable name defined here.   The number of auxiliary variables detected on this line determines the value for naux.  Comments cannot be provided anywhere on this line as they will be interpreted as auxiliary variable names.  Auxiliary variables may not be used by the package, but they will be available for use by other parts of the program.  The program will terminate with an error if auxiliary variables are specified on more than one line in the options block.

\item \texttt{flow\_package\_auxiliary\_name}---keyword to specify the name of an auxiliary variable in the corresponding flow package.  If specified, then the simulated concentrations from this advanced transport package will be copied into the auxiliary variable specified with this name.  Note that the flow package must have an auxiliary variable with this name or the program will terminate with an error.  If the flows for this advanced transport package are read from a file, then this option will have no affect.

\item \texttt{BOUNDNAMES}---keyword to indicate that boundary names may be provided with the list of lake cells.

\item \texttt{PRINT\_INPUT}---keyword to indicate that the list of lake information will be written to the listing file immediately after it is read.

\item \texttt{PRINT\_CONCENTRATION}---keyword to indicate that the list of lake concentration will be printed to the listing file for every stress period in which ``HEAD PRINT'' is specified in Output Control.  If there is no Output Control option and PRINT\_CONCENTRATION is specified, then concentration are printed for the last time step of each stress period.

\item \texttt{PRINT\_FLOWS}---keyword to indicate that the list of lake flow rates will be printed to the listing file for every stress period time step in which ``BUDGET PRINT'' is specified in Output Control.  If there is no Output Control option and ``PRINT\_FLOWS'' is specified, then flow rates are printed for the last time step of each stress period.

\item \texttt{SAVE\_FLOWS}---keyword to indicate that lake flow terms will be written to the file specified with ``BUDGET FILEOUT'' in Output Control.

\item \texttt{CONCENTRATION}---keyword to specify that record corresponds to concentration.

\item \texttt{concfile}---name of the binary output file to write concentration information.

\item \texttt{BUDGET}---keyword to specify that record corresponds to the budget.

\item \texttt{FILEOUT}---keyword to specify that an output filename is expected next.

\item \texttt{budgetfile}---name of the binary output file to write budget information.

\item \texttt{TS6}---keyword to specify that record corresponds to a time-series file.

\item \texttt{FILEIN}---keyword to specify that an input filename is expected next.

\item \texttt{ts6\_filename}---defines a time-series file defining time series that can be used to assign time-varying values. See the ``Time-Variable Input'' section for instructions on using the time-series capability.

\item \texttt{OBS6}---keyword to specify that record corresponds to an observations file.

\item \texttt{obs6\_filename}---name of input file to define observations for the LKT package. See the ``Observation utility'' section for instructions for preparing observation input files. Table \ref{table:obstype} lists observation type(s) supported by the LKT package.

\end{description}
\item \textbf{Block: PACKAGEDATA}

\begin{description}
\item \texttt{lakeno}---integer value that defines the lake number associated with the specified PACKAGEDATA data on the line. LAKENO must be greater than zero and less than or equal to NLAKES. Lake information must be specified for every lake or the program will terminate with an error.  The program will also terminate with an error if information for a lake is specified more than once.

\item \texttt{strt}---real value that defines the starting concentration for the lake.

\item \textcolor{blue}{\texttt{aux}---represents the values of the auxiliary variables for each lake. The values of auxiliary variables must be present for each lake. The values must be specified in the order of the auxiliary variables specified in the OPTIONS block.  If the package supports time series and the Options block includes a TIMESERIESFILE entry (see the ``Time-Variable Input'' section), values can be obtained from a time series by entering the time-series name in place of a numeric value.}

\item \texttt{boundname}---name of the lake cell.  BOUNDNAME is an ASCII character variable that can contain as many as 40 characters.  If BOUNDNAME contains spaces in it, then the entire name must be enclosed within single quotes.

\end{description}
\item \textbf{Block: PERIOD}

\begin{description}
\item \texttt{iper}---integer value specifying the starting stress period number for which the data specified in the PERIOD block apply.  IPER must be less than or equal to NPER in the TDIS Package and greater than zero.  The IPER value assigned to a stress period block must be greater than the IPER value assigned for the previous PERIOD block.  The information specified in the PERIOD block will continue to apply for all subsequent stress periods, unless the program encounters another PERIOD block.

\item \texttt{lakeno}---integer value that defines the lake number associated with the specified PERIOD data on the line. LAKENO must be greater than zero and less than or equal to NLAKES.

\item \texttt{laksetting}---line of information that is parsed into a keyword and values.  Keyword values that can be used to start the LAKSETTING string include: STATUS, CONCENTRATION, RAINFALL, EVAPORATION, RUNOFF, and AUXILIARY.  These settings are used to assign the concentration of associated with the corresponding flow terms.  Concentrations cannot be specified for all flow terms.  For example, the Lake Package supports a ``WITHDRAWAL'' flow term.  If this withdrawal term is active, then water will be withdrawn from the lake at the calculated concentration of the lake.

\begin{lstlisting}[style=blockdefinition]
STATUS <status>
CONCENTRATION <@concentration@>
RAINFALL <@rainfall@>
EVAPORATION <@evaporation@>
RUNOFF <@runoff@>
EXT-INFLOW <@ext-inflow@>
AUXILIARY <auxname> <@auxval@> 
\end{lstlisting}

\item \texttt{status}---keyword option to define lake status.  STATUS can be ACTIVE, INACTIVE, or CONSTANT. By default, STATUS is ACTIVE, which means that concentration will be calculated for the lake.  If a lake is inactive, then there will be no solute mass fluxes into or out of the lake and the inactive value will be written for the lake concentration.  If a lake is constant, then the concentration for the lake will be fixed at the user specified value.

\item \textcolor{blue}{\texttt{concentration}---real or character value that defines the concentration for the lake. The specified CONCENTRATION is only applied if the lake is a constant concentration lake. If the Options block includes a TIMESERIESFILE entry (see the ``Time-Variable Input'' section), values can be obtained from a time series by entering the time-series name in place of a numeric value.}

\item \textcolor{blue}{\texttt{rainfall}---real or character value that defines the rainfall solute concentration $(ML^{-3})$ for the lake. If the Options block includes a TIMESERIESFILE entry (see the ``Time-Variable Input'' section), values can be obtained from a time series by entering the time-series name in place of a numeric value.}

\item \textcolor{blue}{\texttt{evaporation}---real or character value that defines the concentration of evaporated water $(ML^{-3})$ for the lake. If this concentration value is larger than the simulated concentration in the lake, then the evaporated water will be removed at the same concentration as the lake.  If the Options block includes a TIMESERIESFILE entry (see the ``Time-Variable Input'' section), values can be obtained from a time series by entering the time-series name in place of a numeric value.}

\item \textcolor{blue}{\texttt{runoff}---real or character value that defines the concentration of runoff $(ML^{-3})$ for the lake. Value must be greater than or equal to zero. If the Options block includes a TIMESERIESFILE entry (see the ``Time-Variable Input'' section), values can be obtained from a time series by entering the time-series name in place of a numeric value.}

\item \textcolor{blue}{\texttt{ext-inflow}---real or character value that defines the concentration of external inflow $(ML^{-3})$ for the lake. Value must be greater than or equal to zero. If the Options block includes a TIMESERIESFILE entry (see the ``Time-Variable Input'' section), values can be obtained from a time series by entering the time-series name in place of a numeric value.}

\item \texttt{AUXILIARY}---keyword for specifying auxiliary variable.

\item \texttt{auxname}---name for the auxiliary variable to be assigned AUXVAL.  AUXNAME must match one of the auxiliary variable names defined in the OPTIONS block. If AUXNAME does not match one of the auxiliary variable names defined in the OPTIONS block the data are ignored.

\item \textcolor{blue}{\texttt{auxval}---value for the auxiliary variable. If the Options block includes a TIMESERIESFILE entry (see the ``Time-Variable Input'' section), values can be obtained from a time series by entering the time-series name in place of a numeric value.}

\end{description}


\end{description}

\vspace{5mm}
\subsubsection{Example Input File}
\lstinputlisting[style=inputfile]{./mf6ivar/examples/gwt-lkt-example.dat}

\vspace{5mm}
\subsubsection{Available observation types}
Lake Transport Package observations include lake concentration and all of the terms that contribute to the continuity equation for each lake. Additional LKT Package observations include mass flow rates for individual outlets, lakes, or groups of lakes (\texttt{outlet}); the lake volume (\texttt{volume}); lake surface area (\texttt{surface-area}); wetted area for a lake-aquifer connection (\texttt{wetted-area}); and the conductance for a lake-aquifer connection conductance (\texttt{conductance}). The data required for each LKT Package observation type is defined in table~\ref{table:gwt-lktobstype}. Negative and positive values for \texttt{lkt} observations represent a loss from and gain to the GWT model, respectively. For all other flow terms, negative and positive values represent a loss from and gain from the LKT package, respectively.

\begin{longtable}{p{2cm} p{2.75cm} p{2cm} p{1.25cm} p{7cm}}
\caption{Available LKT Package observation types} \tabularnewline

\hline
\hline
\textbf{Stress Package} & \textbf{Observation type} & \textbf{ID} & \textbf{ID2} & \textbf{Description} \\
\hline
\endfirsthead

\captionsetup{textformat=simple}
\caption*{\textbf{Table \arabic{table}.}{\quad}Available LKT Package observation types.---Continued} \tabularnewline

\hline
\hline
\textbf{Stress Package} & \textbf{Observation type} & \textbf{ID} & \textbf{ID2} & \textbf{Description} \\
\hline
\endhead


\hline
\endfoot

% general APT observations
LKT & concentration & lakeno or boundname & -- & Lake concentration. If boundname is specified, boundname must be unique for each lake. \\
LKT & flow-ja-face & lakeno or boundname & lakeno or -- & Mass flow between two lakes connected by an outlet.  If more than one outlet is used to connect the same two lakes, then the mass flow for only the first outlet can be observed.  If a boundname is specified for ID1, then the result is the total mass flow for all outlets for a lake. If a boundname is specified for ID1 then ID2 is not used.\\
LKT & storage & lakeno or boundname & -- & Simulated mass storage flow rate for a lake or group of lakes. \\
LKT & constant & lakeno or boundname & -- & Simulated mass constant-flow rate for a lake or group of lakes. \\
LKT & from-mvr & lakeno or boundname & -- & Simulated mass inflow into a lake or group of lakes from the MVT package. Mass inflow is calculated as the product of provider concentration and the mover flow rate. \\
LKT & to-mvr & outletno or boundname & -- & Mass outflow from a lake outlet, a lake, or a group of lakes that is available for the MVR package. If boundname is not specified for ID, then the outflow available for the MVR package from a specific lake outlet is observed. In this case, ID is the outlet number, which must be between 1 and NOUTLETS. \\
LKT & lkt & lakeno or boundname & \texttt{iconn} or -- & Mass flow rate for a lake or group of lakes and its aquifer connection(s). If boundname is not specified for ID, then the simulated lake-aquifer flow rate at a specific lake connection is observed. In this case, ID2 must be specified and is the connection number \texttt{iconn} for lake \texttt{lakeno}. \\

%observations specific to the lake package
% rainfall evaporation runoff ext-inflow withdrawal outflow
LKT & rainfall & lakeno or boundname & -- & Rainfall rate applied to a lake or group of lakes multiplied by the rainfall concentration. \\
LKT & evaporation & lakeno or boundname & -- & Simulated evaporation rate from a lake or group of lakes multiplied by the evaporation concentration. \\
LKT & runoff & lakeno or boundname & -- & Runoff rate applied to a lake or group of lakes multiplied by the runoff concentration. \\
LKT & ext-inflow & lakeno or boundname & -- & Mass inflow into a lake or group of lakes calculated as the external inflow rate multiplied by the inflow concentration. \\
LKT & withdrawal & lakeno or boundname & -- & Specified withdrawal rate from a lake or group of lakes multiplied by the simulated lake concentration. \\
LKT & ext-outflow & outletno or boundname & -- & External outflow from a lake outlet, a lake, or a group of lakes to an external boundary. If boundname is not specified for ID, then the external outflow from a specific lake outlet is observed. In this case, ID is the outlet number outletno.

%LKT & outlet-inflow & lakeno or boundname & -- & Simulated inflow from upstream lake outlets into a lake or group of lakes. \\
%LKT & inflow & lakeno or boundname & -- & Sum of specified inflow and simulated inflow from upstream lake outlets into a lake or group of lakes. \\
%LKT & outlet & outletno or boundname & -- & Simulate outlet flow rate from a lake outlet, a lake, or a group of lakes. If boundname is not specified for ID, then the flow from a specific lake outlet is observed. In this case, ID is the outlet number outletno. \\
%LKT & volume & lakeno or boundname & -- & Simulated lake volume or group of lakes. \\
%LKT & surface-area & lakeno or boundname & -- & Simulated surface area for a lake or group of lakes. \\
%LKT & wetted-area & lakeno or boundname & \texttt{iconn} or -- & Simulated wetted-area for a lake or group of lakes and its aquifer connection(s). If boundname is not specified for ID, then the wetted area of a specific lake connection is observed. In this case, ID2 must be specified and is the connection number \texttt{iconn}. \\
%LKT & conductance & lakeno or boundname & \texttt{iconn} or -- & Calculated conductance for a lake or group of lakes and its aquifer connection(s). If boundname is not specified for ID, then the calculated conductance of a specific lake connection is observed. In this case, ID2 must be specified and is the connection number \texttt{iconn}.

\label{table:gwt-lktobstype}
\end{longtable}

\vspace{5mm}
\subsubsection{Example Observation Input File}
\lstinputlisting[style=inputfile]{./mf6ivar/examples/gwt-lkt-example-obs.dat}


