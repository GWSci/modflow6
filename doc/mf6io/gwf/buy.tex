Input to the Buoyancy (BUY) Package is read from the file that has type ``BUY6'' in the Name File.  If the BUY Package is included for a model, then the model will use the variable-density form of Darcy's Law for all flow calculations using the approach described by \cite{langevin2020hydraulic}.  Only one BUY Package can be specified for a GWF model. The BUY Package can be coupled with one or more GWT Models so that fluid density is updated dynamically with one or more simulated concentration fields.

The BUY Package calculates fluid density using the following equation of state from \cite{langevin2008seawat}:

\begin{equation}
\label{eqn:volumeconservationdiscrete}
%\rho = \rho_0 + \sum_{i=1}^{NRHOSPECIES} \left ( C_i - C_{i,0} \right )
\rho = DENSEREF + \sum_{i=1}^{NRHOSPECIES} DRHODC_i \left ( CONCENTRATION_i - CRHOREF_i \right )
\end{equation}

\noindent where $\rho$ is the calculated density, $DENSEREF$ is the density of a reference fluid, typically taken to be freshwater at a temperature of 25 degrees Celsius; $NRHOSPECIES$ is the number of chemical species that contribute to the density calculation, $DRHODC_i$ is the parameter that describes how density changes as a function of concentration for chemical species $i$ (i.e. the slope of a line that relates density to concentration), $CONCENTRATION_i$ is the concentration of species $i$, and $CRHOREF_i$ is the concentration of species $i$ in the reference fluid, which is normally set to zero.

\subsubsection{Stress Packages}
For head-dependent stress packages, the BUY Package may require fluid density and elevation for each head-dependent boundary so that the model can use a variable-density form of Darcy's Law to calculate flow between the boundary and the aquifer.  By default, the boundary elevation is set equal to the cell elevation.  For water-table conditions, the cell elevation is calculated as bottom elevation plus half of saturation multiplied by the cell thickness.  If desired, the user can more precisely locate the boundary elevation by specifying an auxiliary variable with the name ``ELEVATION''.  The program will use the values in this column as the boundary elevation.  A situation where this may be required is for river or general-head boundaries that are conceptualized as being on top of a model cell.  In those cases, an ELEVATION column should be specified and the values set to the top of the cell or some other appropriate elevation that corresponds to where the boundary stage applies.

By default, the boundary density is set equal to DENSEREF, commonly specified as the density of freshwater; however, there are two other options for setting the density of a boundary package.  The first is to assign an auxiliary variable with the name ``DENSITY''.  If this auxiliary variable is detected, then the density value in this column will be assigned to the density for the boundary.  Alternatively, a density value can be calculated for each boundary using the density equation of state and one or more concentrations provided as auxiliary variables.  In this case, the user must assign one auxiliary variable for each AUXSPECIESNAME listed in the PACKAGEDATA block below.  Thus, there must be NRHOSPECIES auxiliary variables, each with the identical name as those specified in PACKAGEDATA.  The BUY Package will calculate the density for each boundary using these concentrations and the values specified for DENSEREF, DRHODC, and CRHOREF.  If the boundary package contains an auxiliary variable named DENSITY and also contains AUXSPECIESNAME auxiliary variables, then the boundary density value will be assigned to the one in the DENSITY auxiliary variable.

A GWT Model can be used to calculate concentrations for the advanced stress packages (LAK, SFR, MAW, and UZF) if corresponding advanced transport packages are specified (LKT, SFT, MWT, and UZT).  The advanced stress packages have an input option called FLOW\_PACKAGE\_AUXILIARY\_NAME.  When activated, this option will result in the simulated concentration for a lake or other feature being copied from the advanced transport package into the auxiliary variable for the corresponding GWF stress package.  This means that the density for a lake or stream, for example, can be dynamically updated during the simulation using concentrations from advanced transport packages that are fed into auxiliary variables in the advanced stress packages, and ultimately used by the BUY Package to calculate a fluid density using the equation of state.  This concept also applies when multiple GWT Models are used simultaneously to simulate multiple species.  In this case, multiple auxiliary variables are required for an advanced stress package, with each one representing a concentration from a different GWT Model.  

\begin{longtable}{p{3cm} p{12cm}}
\caption{Description of density terms for stress packages}
\tabularnewline
\hline
\hline
\textbf{Stress Package} & \textbf{Note} \\
\hline
\endhead
\hline
\endfoot
GHB & ELEVATION can be specified as an auxiliary variable.  A DENSITY auxiliary variable or one or more auxiliary variables for calculating density in the equation of state can be specified \\
RIV & ELEVATION can be specified as an auxiliary variable.  A DENSITY auxiliary variable or one or more auxiliary variables for calculating density in the equation of state can be specified \\
DRN & The drain formulation assumes that the drain boundary contains water of the same density as the discharging water; auxiliary variables have no affect on the drain calculation  \\
LAK & Elevation for each lake-aquifer connection is determined based on lake bottom and adjacent cell elevations. A DENSITY auxiliary variable or one or more auxiliary variables for calculating density in the equation of state can be specified \\
SFR & Elevation for each sfr-aquifer connection is determined based on stream bottom and adjacent cell elevations. A DENSITY auxiliary variable or one or more auxiliary variables for calculating density in the equation of state can be specified \\
MAW & Elevation for each maw-aquifer connection is determined based on cell elevation. A DENSITY auxiliary variable or one or more auxiliary variables for calculating density in the equation of state can be specified \\
UZF & No density terms implemented \\
\end{longtable}

\vspace{5mm}
\subsubsection{Structure of Blocks}

\vspace{5mm}
\noindent \textit{FOR EACH SIMULATION}
\lstinputlisting[style=blockdefinition]{./mf6ivar/tex/gwf-buy-options.dat}
\lstinputlisting[style=blockdefinition]{./mf6ivar/tex/gwf-buy-dimensions.dat}
\lstinputlisting[style=blockdefinition]{./mf6ivar/tex/gwf-buy-packagedata.dat}
%\vspace{5mm}
%\noindent \textit{FOR ANY STRESS PERIOD}
%\lstinputlisting[style=blockdefinition]{./mf6ivar/tex/gwf-buy-period.dat}

\vspace{5mm}
\subsubsection{Explanation of Variables}
\begin{description}
% DO NOT MODIFY THIS FILE DIRECTLY.  IT IS CREATED BY mf6ivar.py 

\item \textbf{Block: OPTIONS}

\begin{description}
\item \texttt{HHFORMULATION\_RHS}---use the variable-density hydraulic head formulation and add off-diagonal terms to the right-hand.  This option will prevent the BUY Package from adding asymmetric terms to the flow matrix.

\item \texttt{denseref}---fluid reference density used in the equation of state.  This value is set to 1000. if not specified as an option.

\item \texttt{DENSITY}---keyword to specify that record corresponds to density.

\item \texttt{FILEOUT}---keyword to specify that an output filename is expected next.

\item \texttt{densityfile}---name of the binary output file to write density information.  The density file has the same format as the head file.  Density values will be written to the density file whenever heads are written to the binary head file.  The settings for controlling head output are contained in the Output Control option.

\end{description}
\item \textbf{Block: DIMENSIONS}

\begin{description}
\item \texttt{nrhospecies}---number of species used in density equation of state.  This value must be one or greater.  The value must be one if concentrations are specified using the CONCENTRATION keyword in the PERIOD block below.

\end{description}
\item \textbf{Block: PACKAGEDATA}

\begin{description}
\item \texttt{irhospec}---integer value that defines the species number associated with the specified PACKAGEDATA data on the line. IRHOSPECIES must be greater than zero and less than or equal to NRHOSPECIES. Information must be specified for each of the NRHOSPECIES species or the program will terminate with an error.  The program will also terminate with an error if information for a species is specified more than once.

\item \texttt{drhodc}---real value that defines the slope of the density-concentration line for this species used in the density equation of state.

\item \texttt{crhoref}---real value that defines the reference concentration value used for this species in the density equation of state.

\item \texttt{modelname}---name of GWT model used to simulate a species that will be used in the density equation of state.  This name will have no affect if the simulation does not include a GWT model that corresponds to this GWF model.

\item \texttt{auxspeciesname}---name of an auxiliary variable in a GWF stress package that will be used for this species to calculate a density value.  If a density value is needed by the Buoyancy Package then it will use the concentration values in this AUXSPECIESNAME column in the density equation of state.  For advanced stress packages (LAK, SFR, MAW, and UZF) that have an associated advanced transport package (LKT, SFT, MWT, and UZT), the FLOW\_PACKAGE\_AUXILIARY\_NAME option in the advanced transport package can be used to transfer simulated concentrations into the flow package auxiliary variable.  In this manner, the Buoyancy Package can calculate density values for lakes, streams, multi-aquifer wells, and unsaturated zone flow cells using simulated concentrations.

\end{description}
\item \textbf{Block: PERIOD}

\begin{description}
\item \texttt{iper}---integer value specifying the starting stress period number for which the data specified in the PERIOD block apply.  IPER must be less than or equal to NPER in the TDIS Package and greater than zero.  The IPER value assigned to a stress period block must be greater than the IPER value assigned for the previous PERIOD block.  The information specified in the PERIOD block will continue to apply for all subsequent stress periods, unless the program encounters another PERIOD block.

\item \texttt{elevation}---cell center elevation for each cell in the grid.  This may be useful for constructing two-dimensional cross section models using an areal grid.

\item \texttt{concentration}---user-specified concentration array.  If this array is specified, then these concentrations are converted to densities using the equation of state values specified in the PACKAGEDATA block.  If this CONCENTRATION array is specified, then the NRHOSPECIES dimension must be one.

\end{description}


\end{description}

\vspace{5mm}
\subsubsection{Example Input File}
\lstinputlisting[style=inputfile]{./mf6ivar/examples/gwf-buy-example.dat}

