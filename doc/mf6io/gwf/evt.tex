Input to the Evapotranspiration (EVT) Package is read from the file that has type ``EVT6'' in the Name File. Any number of EVT Packages can be specified for a single groundwater flow model. All single-valued variables are free format.

Evapotranspiration input can be specified using lists or arrays, unless the DISU Package is used.  List-based input must be used if discretization is specified using the DISU Package.  List-based input for evapotranspiration is the default, and is described here.  Instructions for specifying array-based evapotranspiration are described in the next section. 

List-based input offers several advantages over the array-based input for specifying evapotranspiration.  First, multiple list entries can be specified for a single cell.  This makes it possible to divide a cell into multiple areas, and assign a different evapotranspiration rate or extinction depth for each area (perhaps based on vegetation type or some other criteria).  In this case, the user would likely specify an auxiliary variable to serve as a multiplier.  This multiplier would be calculated by the user and provided in the input file as the fractional cell are for the individual evapotranspiration entries.  Another advantage to using list-based input for specifying evapotranspiration is that boundnames can be specified.  Boundnames work with the Observations capability and can be used to sum evapotranspiration rates for entries with the same boundname.  A disadvantage of the list-based input is that one cannot easily assign evapotranspiration to the entire model without specifying a list of model cells.  For this reason \mf also supports array-based input for evapotranspiration.

ET input is read in list form, as shown in the PERIOD block below. Each line in the PERIOD block defines all input for one cell. Entries following \texttt{cellid}, in order, define the ET surface (\texttt{etss}), maximum ET flux rate (\texttt{etsr}), extinction depth (\texttt{etsx}), all (\texttt{netseg} -- 1) \texttt{pxdp} values, all (\texttt{netseg} -- 1) \texttt{petm} values, all auxiliary variables (if AUXILIARY option is specified), and boundary name (if BOUNDNAMES option is specified).

\vspace{5mm}
\subsubsection{Structure of Blocks}
\vspace{5mm}

\noindent \textit{FOR EACH SIMULATION}
\lstinputlisting[style=blockdefinition]{./mf6ivar/tex/gwf-evt-options.dat}
\lstinputlisting[style=blockdefinition]{./mf6ivar/tex/gwf-evt-dimensions.dat}
\vspace{5mm}
\noindent \textit{FOR ANY STRESS PERIOD}
\lstinputlisting[style=blockdefinition]{./mf6ivar/tex/gwf-evt-period.dat}
\packageperioddescription

\vspace{5mm}
\subsubsection{Explanation of Variables}
\begin{description}
% DO NOT MODIFY THIS FILE DIRECTLY.  IT IS CREATED BY mf6ivar.py 

\item \textbf{Block: OPTIONS}

\begin{description}
\item \texttt{FIXED\_CELL}---indicates that evapotranspiration will not be reassigned to a cell underlying the cell specified in the list if the specified cell is inactive.

\item \texttt{auxiliary}---defines an array of one or more auxiliary variable names.  There is no limit on the number of auxiliary variables that can be provided on this line; however, lists of information provided in subsequent blocks must have a column of data for each auxiliary variable name defined here.   The number of auxiliary variables detected on this line determines the value for naux.  Comments cannot be provided anywhere on this line as they will be interpreted as auxiliary variable names.  Auxiliary variables may not be used by the package, but they will be available for use by other parts of the program.  The program will terminate with an error if auxiliary variables are specified on more than one line in the options block.

\item \texttt{auxmultname}---name of auxiliary variable to be used as multiplier of evapotranspiration rate.

\item \texttt{BOUNDNAMES}---keyword to indicate that boundary names may be provided with the list of evapotranspiration cells.

\item \texttt{PRINT\_INPUT}---keyword to indicate that the list of evapotranspiration information will be written to the listing file immediately after it is read.

\item \texttt{PRINT\_FLOWS}---keyword to indicate that the list of evapotranspiration flow rates will be printed to the listing file for every stress period time step in which ``BUDGET PRINT'' is specified in Output Control.  If there is no Output Control option and ``PRINT\_FLOWS'' is specified, then flow rates are printed for the last time step of each stress period.

\item \texttt{SAVE\_FLOWS}---keyword to indicate that evapotranspiration flow terms will be written to the file specified with ``BUDGET FILEOUT'' in Output Control.

\item \texttt{TS6}---keyword to specify that record corresponds to a time-series file.

\item \texttt{FILEIN}---keyword to specify that an input filename is expected next.

\item \texttt{ts6\_filename}---defines a time-series file defining time series that can be used to assign time-varying values. See the ``Time-Variable Input'' section for instructions on using the time-series capability.

\item \texttt{OBS6}---keyword to specify that record corresponds to an observations file.

\item \texttt{obs6\_filename}---name of input file to define observations for the Evapotranspiration package. See the ``Observation utility'' section for instructions for preparing observation input files. Table \ref{table:obstype} lists observation type(s) supported by the Evapotranspiration package.

\item \texttt{SURF\_RATE\_SPECIFIED}---indicates that the proportion of the evapotranspiration rate at the ET surface will be specified as PETM0 in list input.

\end{description}
\item \textbf{Block: DIMENSIONS}

\begin{description}
\item \texttt{maxbound}---integer value specifying the maximum number of evapotranspiration cells cells that will be specified for use during any stress period.

\item \texttt{nseg}---number of ET segments.  Default is one.  When NSEG is greater than 1, PXDP and PETM arrays must be specified NSEG - 1 times each, in order from the uppermost segment down. PXDP defines the extinction-depth proportion at the bottom of a segment. PETM defines the proportion of the maximum ET flux rate at the bottom of a segment.

\end{description}
\item \textbf{Block: PERIOD}

\begin{description}
\item \texttt{iper}---integer value specifying the starting stress period number for which the data specified in the PERIOD block apply.  IPER must be less than or equal to NPER in the TDIS Package and greater than zero.  The IPER value assigned to a stress period block must be greater than the IPER value assigned for the previous PERIOD block.  The information specified in the PERIOD block will continue to apply for all subsequent stress periods, unless the program encounters another PERIOD block.

\item \texttt{cellid}---is the cell identifier, and depends on the type of grid that is used for the simulation.  For a structured grid that uses the DIS input file, CELLID is the layer, row, and column.   For a grid that uses the DISV input file, CELLID is the layer and CELL2D number.  If the model uses the unstructured discretization (DISU) input file or the linear discretization with vertices (DISL), CELLID is the node number for the cell.

\item \textcolor{blue}{\texttt{surface}---is the elevation of the ET surface ($L$). If the Options block includes a TIMESERIESFILE entry (see the ``Time-Variable Input'' section), values can be obtained from a time series by entering the time-series name in place of a numeric value.}

\item \textcolor{blue}{\texttt{rate}---is the maximum ET flux rate ($LT^{-1}$). If the Options block includes a TIMESERIESFILE entry (see the ``Time-Variable Input'' section), values can be obtained from a time series by entering the time-series name in place of a numeric value.}

\item \textcolor{blue}{\texttt{depth}---is the ET extinction depth ($L$). If the Options block includes a TIMESERIESFILE entry (see the ``Time-Variable Input'' section), values can be obtained from a time series by entering the time-series name in place of a numeric value.}

\item \textcolor{blue}{\texttt{pxdp}---is the proportion of the ET extinction depth at the bottom of a segment (dimensionless). If the Options block includes a TIMESERIESFILE entry (see the ``Time-Variable Input'' section), values can be obtained from a time series by entering the time-series name in place of a numeric value.}

\item \textcolor{blue}{\texttt{petm}---is the proportion of the maximum ET flux rate at the bottom of a segment (dimensionless). If the Options block includes a TIMESERIESFILE entry (see the ``Time-Variable Input'' section), values can be obtained from a time series by entering the time-series name in place of a numeric value.}

\item \textcolor{blue}{\texttt{petm0}---is the proportion of the maximum ET flux rate that will apply when head is at or above the ET surface (dimensionless). PETM0 is read only when the SURF\_RATE\_SPECIFIED option is used. If the Options block includes a TIMESERIESFILE entry (see the ``Time-Variable Input'' section), values can be obtained from a time series by entering the time-series name in place of a numeric value.}

\item \textcolor{blue}{\texttt{aux}---represents the values of the auxiliary variables for each evapotranspiration. The values of auxiliary variables must be present for each evapotranspiration. The values must be specified in the order of the auxiliary variables specified in the OPTIONS block.  If the package supports time series and the Options block includes a TIMESERIESFILE entry (see the ``Time-Variable Input'' section), values can be obtained from a time series by entering the time-series name in place of a numeric value.}

\item \texttt{boundname}---name of the evapotranspiration cell.  BOUNDNAME is an ASCII character variable that can contain as many as 40 characters.  If BOUNDNAME contains spaces in it, then the entire name must be enclosed within single quotes.

\end{description}


\end{description}

\vspace{5mm}
\subsubsection{Example Input File}
\lstinputlisting[style=inputfile]{./mf6ivar/examples/gwf-evt-example.dat}

\vspace{5mm}
\subsubsection{Available observation types}
Evapotranspiration Package observations are limited to the simulated evapotranspiration flow rates (\texttt{evt}). The data required for the EVT Package observation type is defined in table~\ref{table:gwf-evtobstype}. Negative and positive values for an observation represent a loss from and gain to the GWF model, respectively.

\begin{longtable}{p{2cm} p{2.75cm} p{2cm} p{1.25cm} p{7cm}}
\caption{Available EVT Package observation types} \tabularnewline

\hline
\hline
\textbf{Stress Package} & \textbf{Observation type} & \textbf{ID} & \textbf{ID2} & \textbf{Description} \\
\hline
\endhead

\hline
\endfoot

EVT & evt & cellid or boundname & -- & Flow from the groundwater system through an evapotranspiration boundary or group of evapotranspiration boundaries.
\label{table:gwf-evtobstype}
\end{longtable}

\vspace{5mm}
\subsubsection{Example Observation Input File}
\lstinputlisting[style=inputfile]{./mf6ivar/examples/gwf-evt-example-obs.dat}
