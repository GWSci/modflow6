% DO NOT MODIFY THIS FILE DIRECTLY.  IT IS CREATED BY mf6ivar.py 

\item \textbf{Block: OPTIONS}

\begin{description}
\item \texttt{auxiliary}---defines an array of one or more auxiliary variable names.  There is no limit on the number of auxiliary variables that can be provided on this line; however, lists of information provided in subsequent blocks must have a column of data for each auxiliary variable name defined here.   The number of auxiliary variables detected on this line determines the value for naux.  Comments cannot be provided anywhere on this line as they will be interpreted as auxiliary variable names.  Auxiliary variables may not be used by the package, but they will be available for use by other parts of the program.  The program will terminate with an error if auxiliary variables are specified on more than one line in the options block.

\item \texttt{BOUNDNAMES}---keyword to indicate that boundary names may be provided with the list of multi-aquifer well cells.

\item \texttt{PRINT\_INPUT}---keyword to indicate that the list of multi-aquifer well information will be written to the listing file immediately after it is read.

\item \texttt{PRINT\_HEAD}---keyword to indicate that the list of multi-aquifer well heads will be printed to the listing file for every stress period in which ``HEAD PRINT'' is specified in Output Control.  If there is no Output Control option and PRINT\_HEAD is specified, then heads are printed for the last time step of each stress period.

\item \texttt{PRINT\_FLOWS}---keyword to indicate that the list of multi-aquifer well flow rates will be printed to the listing file for every stress period time step in which ``BUDGET PRINT'' is specified in Output Control.  If there is no Output Control option and ``PRINT\_FLOWS'' is specified, then flow rates are printed for the last time step of each stress period.

\item \texttt{SAVE\_FLOWS}---keyword to indicate that multi-aquifer well flow terms will be written to the file specified with ``BUDGET FILEOUT'' in Output Control.

\item \texttt{HEAD}---keyword to specify that record corresponds to head.

\item \texttt{headfile}---name of the binary output file to write stage information.

\item \texttt{BUDGET}---keyword to specify that record corresponds to the budget.

\item \texttt{FILEOUT}---keyword to specify that an output filename is expected next.

\item \texttt{budgetfile}---name of the binary output file to write budget information.

\item \texttt{NO\_WELL\_STORAGE}---keyword that deactivates inclusion of well storage contributions to the multi-aquifer well package continuity equation.

\item \texttt{FLOWING\_WELLS}---keyword that activates the flowing wells option for the multi-aquifer well package.

\item \texttt{shutdown\_theta}---value that defines the weight applied to discharge rate for wells that limit the water level in a discharging well (defined using the HEAD\_LIMIT keyword in the stress period data). SHUTDOWN\_THETA is used to control discharge rate oscillations when the flow rate from the aquifer is less than the specified flow rate from the aquifer to the well. Values range between 0.0 and 1.0, and larger values increase the weight (decrease under-relaxation) applied to the well discharge rate. The HEAD\_LIMIT option has been included to facilitate backward compatibility with previous versions of MODFLOW but use of the RATE\_SCALING option instead of the HEAD\_LIMIT option is recommended. By default, SHUTDOWN\_THETA is 0.7.

\item \texttt{shutdown\_kappa}---value that defines the weight applied to discharge rate for wells that limit the water level in a discharging well (defined using the HEAD\_LIMIT keyword in the stress period data). SHUTDOWN\_KAPPA is used to control discharge rate oscillations when the flow rate from the aquifer is less than the specified flow rate from the aquifer to the well. Values range between 0.0 and 1.0, and larger values increase the weight applied to the well discharge rate. The HEAD\_LIMIT option has been included to facilitate backward compatibility with previous versions of MODFLOW but use of the RATE\_SCALING option instead of the HEAD\_LIMIT option is recommended. By default, SHUTDOWN\_KAPPA is 0.0001.

\item \texttt{TS6}---keyword to specify that record corresponds to a time-series file.

\item \texttt{FILEIN}---keyword to specify that an input filename is expected next.

\item \texttt{ts6\_filename}---defines a time-series file defining time series that can be used to assign time-varying values. See the ``Time-Variable Input'' section for instructions on using the time-series capability.

\item \texttt{OBS6}---keyword to specify that record corresponds to an observations file.

\item \texttt{obs6\_filename}---name of input file to define observations for the MAW package. See the ``Observation utility'' section for instructions for preparing observation input files. Table \ref{table:obstype} lists observation type(s) supported by the MAW package.

\item \texttt{MOVER}---keyword to indicate that this instance of the MAW Package can be used with the Water Mover (MVR) Package.  When the MOVER option is specified, additional memory is allocated within the package to store the available, provided, and received water.

\end{description}
\item \textbf{Block: DIMENSIONS}

\begin{description}
\item \texttt{nmawwells}---integer value specifying the number of multi-aquifer wells that will be simulated for all stress periods.

\end{description}
\item \textbf{Block: PACKAGEDATA}

\begin{description}
\item \texttt{wellno}---integer value that defines the well number associated with the specified PACKAGEDATA data on the line. WELLNO must be greater than zero and less than or equal to NMAWWELLS. Multi-aquifer well information must be specified for every multi-aquifer well or the program will terminate with an error.  The program will also terminate with an error if information for a multi-aquifer well is specified more than once.

\item \texttt{radius}---radius for the multi-aquifer well.

\item \texttt{bottom}---bottom elevation of the multi-aquifer well. The well bottom is reset to the cell bottom in the lowermost GWF cell connection in cases where the specified well bottom is above the bottom of this GWF cell.

\item \texttt{strt}---starting head for the multi-aquifer well.

\item \texttt{condeqn}---character string that defines the conductance equation that is used to calculate the saturated conductance for the multi-aquifer well. Possible multi-aquifer well CONDEQN strings include: SPECIFIED--character keyword to indicate the multi-aquifer well saturated conductance will be specified.  THIEM--character keyword to indicate the multi-aquifer well saturated conductance will be calculated using the Thiem equation, which considers the cell top and bottom, aquifer hydraulic conductivity, and effective cell and well radius.  SKIN--character keyword to indicate that the multi-aquifer well saturated conductance will be calculated using the cell top and bottom, aquifer and screen hydraulic conductivity, and well and skin radius.  CUMULATIVE--character keyword to indicate that the multi-aquifer well saturated conductance will be calculated using a combination of the Thiem and SKIN equations.  MEAN--character keyword to indicate the multi-aquifer well saturated conductance will be calculated using the aquifer and screen top and bottom, aquifer and screen hydraulic conductivity, and well and skin radius. The CUMULATIVE conductance equation is identical to the SKIN LOSSTYPE in the Multi-Node Well (MNW2) package for MODFLOW-2005. The program will terminate with an error condition if CONDEQN is SKIN or CUMULATIVE and the calculated saturated conductance is less than zero; if an error condition occurs, it is suggested that the THEIM or MEAN conductance equations be used for these multi-aquifer wells.

\item \texttt{ngwfnodes}---integer value that defines the number of GWF nodes connected to this (WELLNO) multi-aquifer well. NGWFNODES must be greater than zero.

\item \textcolor{blue}{\texttt{aux}---represents the values of the auxiliary variables for each multi-aquifer well. The values of auxiliary variables must be present for each multi-aquifer well. The values must be specified in the order of the auxiliary variables specified in the OPTIONS block.  If the package supports time series and the Options block includes a TIMESERIESFILE entry (see the ``Time-Variable Input'' section), values can be obtained from a time series by entering the time-series name in place of a numeric value.}

\item \texttt{boundname}---name of the multi-aquifer well cell.  BOUNDNAME is an ASCII character variable that can contain as many as 40 characters.  If BOUNDNAME contains spaces in it, then the entire name must be enclosed within single quotes.

\end{description}
\item \textbf{Block: CONNECTIONDATA}

\begin{description}
\item \texttt{wellno}---integer value that defines the well number associated with the specified CONNECTIONDATA data on the line. WELLNO must be greater than zero and less than or equal to NMAWWELLS. Multi-aquifer well connection information must be specified for every multi-aquifer well connection to the GWF model (NGWFNODES) or the program will terminate with an error.  The program will also terminate with an error if connection information for a multi-aquifer well connection to the GWF model is specified more than once.

\item \texttt{icon}---integer value that defines the GWF connection number for this multi-aquifer well connection entry. ICONN must be greater than zero and less than or equal to NGWFNODES for multi-aquifer well WELLNO.

\item \texttt{cellid}---is the cell identifier, and depends on the type of grid that is used for the simulation.  For a structured grid that uses the DIS input file, CELLID is the layer, row, and column.   For a grid that uses the DISV input file, CELLID is the layer and CELL2D number.  If the model uses the unstructured discretization (DISU) input file, CELLID is the node number for the cell. One or more screened intervals can be connected to the same CELLID if CONDEQN for a well is MEAN. The program will terminate with an error if MAW wells using SPECIFIED, THIEM, SKIN, or CUMULATIVE conductance equations have more than one connection to the same CELLID.

\item \texttt{scrn\_top}---value that defines the top elevation of the screen for the multi-aquifer well connection. If the specified SCRN\_TOP is greater than the top of the GWF cell it is set equal to the top of the cell.  SCRN\_TOP can be any value if CONDEQN is SPECIFIED, THIEM, SKIN, or COMPOSITE and SCRN\_TOP is set to the top of the cell.

\item \texttt{scrn\_bot}---value that defines the bottom elevation of the screen for the multi-aquifer well connection. If the specified SCRN\_BOT is less than the bottom of the GWF cell it is set equal to the bottom of the cell.  SCRN\_BOT can be any value if CONDEQN is SPECIFIED, THIEM, SKIN, or COMPOSITE and SCRN\_BOT is set to the bottom of the cell.

\item \texttt{hk\_skin}---value that defines the skin (filter pack) hydraulic conductivity (if CONDEQN for the multi-aquifer well is SKIN, CUMULATIVE, or MEAN) or conductance (if CONDEQN for the multi-aquifer well is SPECIFIED) for each GWF node connected to the multi-aquifer well (NGWFNODES). If CONDEQN is SPECIFIED, HK\_SKIN must be greater than or equal to zero.  HK\_SKIN can be any value if CONDEQN is THIEM. Otherwise, HK\_SKIN must be greater than zero. If CONDEQN is SKIN, the contrast between the cell transmissivity (the product of geometric mean horizontal hydraulic conductivity and the cell thickness) and the well transmissivity (the product of HK\_SKIN and the screen thicknesses) must be greater than one in node CELLID or the program will terminate with an error condition; if an error condition occurs, it is suggested that the HK\_SKIN be reduced to a value less than K11 and K22 in node CELLID or the THEIM or MEAN conductance equations be used for these multi-aquifer wells.

\item \texttt{radius\_skin}---real value that defines the skin radius (filter pack radius) for the multi-aquifer well. RADIUS\_SKIN can be any value if CONDEQN is SPECIFIED or THIEM. Otherwise, RADIUS\_SKIN must be greater than RADIUS for the multi-aquifer well.

\end{description}
\item \textbf{Block: PERIOD}

\begin{description}
\item \texttt{iper}---integer value specifying the starting stress period number for which the data specified in the PERIOD block apply.  IPER must be less than or equal to NPER in the TDIS Package and greater than zero.  The IPER value assigned to a stress period block must be greater than the IPER value assigned for the previous PERIOD block.  The information specified in the PERIOD block will continue to apply for all subsequent stress periods, unless the program encounters another PERIOD block.

\item \texttt{wellno}---integer value that defines the well number associated with the specified PERIOD data on the line. WELLNO must be greater than zero and less than or equal to NMAWWELLS.

\item \texttt{mawsetting}---line of information that is parsed into a keyword and values.  Keyword values that can be used to start the MAWSETTING string include: STATUS, FLOWING\_WELL, RATE, WELL\_HEAD, HEAD\_LIMIT, SHUT\_OFF, RATE\_SCALING, and AUXILIARY.

\begin{lstlisting}[style=blockdefinition]
STATUS <status>
FLOWING_WELL <fwelev> <fwcond> <fwrlen> 
RATE <@rate@>
WELL_HEAD <@well_head@>
HEAD_LIMIT <head_limit>
SHUT_OFF <minrate> <maxrate> 
RATE_SCALING <pump_elevation> <scaling_length> 
AUXILIARY <auxname> <@auxval@> 
\end{lstlisting}

\item \texttt{status}---keyword option to define well status.  STATUS can be ACTIVE, INACTIVE, or CONSTANT. By default, STATUS is ACTIVE.

\item \texttt{FLOWING\_WELL}---keyword to indicate the well is a flowing well.  The FLOWING\_WELL option can be used to simulate flowing wells when the simulated well head exceeds the specified drainage elevation.

\item \texttt{fwelev}---elevation used to determine whether or not the well is flowing.

\item \texttt{fwcond}---conductance used to calculate the discharge of a free flowing well.  Flow occurs when the head in the well is above the well top elevation (FWELEV).

\item \texttt{fwrlen}---length used to reduce the conductance of the flowing well.  When the head in the well drops below the well top plus the reduction length, then the conductance is reduced.  This reduction length can be used to improve the stability of simulations with flowing wells so that there is not an abrupt change in flowing well rates.

\item \textcolor{blue}{\texttt{rate}---is the volumetric pumping rate for the multi-aquifer well. A positive value indicates recharge and a negative value indicates discharge (pumping). RATE only applies to active (IBOUND $>$ 0) multi-aquifer wells. If the Options block includes a TIMESERIESFILE entry (see the ``Time-Variable Input'' section), values can be obtained from a time series by entering the time-series name in place of a numeric value. By default, the RATE for each multi-aquifer well is zero.}

\item \textcolor{blue}{\texttt{well\_head}---is the head in the multi-aquifer well. WELL\_HEAD is only applied to constant head (STATUS is CONSTANT) and inactive (STATUS is INACTIVE) multi-aquifer wells. If the Options block includes a TIMESERIESFILE entry (see the ``Time-Variable Input'' section), values can be obtained from a time series by entering the time-series name in place of a numeric value.}

\item \texttt{head\_limit}---is the limiting water level (head) in the well, which is the minimum of the well RATE or the well inflow rate from the aquifer. HEAD\_LIMIT can be applied to extraction wells (RATE $<$ 0) or injection wells (RATE $>$ 0). HEAD\_LIMIT can be deactivated by specifying the text string `OFF'. The HEAD\_LIMIT option is based on the HEAD\_LIMIT functionality available in the MNW2~\citep{konikow2009} package for MODFLOW-2005. The HEAD\_LIMIT option has been included to facilitate backward compatibility with previous versions of MODFLOW but use of the RATE\_SCALING option instead of the HEAD\_LIMIT option is recommended. By default, HEAD\_LIMIT is `OFF'.

\item \texttt{SHUT\_OFF}---keyword for activating well shut off capability.  Subsequent values define the minimum and maximum pumping rate that a well must exceed to shutoff or reactivate a well, respectively, during a stress period. SHUT\_OFF is only applied to injection wells (RATE$<0$) and if HEAD\_LIMIT is specified (not set to `OFF').  If HEAD\_LIMIT is specified, SHUT\_OFF can be deactivated by specifying a minimum value equal to zero. The SHUT\_OFF option is based on the SHUT\_OFF functionality available in the MNW2~\citep{konikow2009} package for MODFLOW-2005. The SHUT\_OFF option has been included to facilitate backward compatibility with previous versions of MODFLOW but use of the RATE\_SCALING option instead of the SHUT\_OFF option is recommended. By default, SHUT\_OFF is not used.

\item \texttt{minrate}---is the minimum rate that a well must exceed to shutoff a well during a stress period. The well will shut down during a time step if the flow rate to the well from the aquifer is less than MINRATE. If a well is shut down during a time step, reactivation of the well cannot occur until the next time step to reduce oscillations. MINRATE must be less than maxrate.

\item \texttt{maxrate}---is the maximum rate that a well must exceed to reactivate a well during a stress period. The well will reactivate during a timestep if the well was shutdown during the previous time step and the flow rate to the well from the aquifer exceeds maxrate. Reactivation of the well cannot occur until the next time step if a well is shutdown to reduce oscillations. maxrate must be greater than MINRATE.

\item \texttt{RATE\_SCALING}---activate rate scaling.  If RATE\_SCALING is specified, both PUMP\_ELEVATION and SCALING\_LENGTH must be specified. RATE\_SCALING cannot be used with HEAD\_LIMIT.  RATE\_SCALING can be used for extraction or injection wells.  For extraction wells, the extraction rate will start to decrease once the head in the well lowers to a level equal to the pump elevation plus the scaling length.  If the head in the well drops below the pump elevation, then the extraction rate is calculated to be zero.  For an injection well, the injection rate will begin to decrease once the head in the well rises above the specified pump elevation.  If the head in the well rises above the pump elevation plus the scaling length, then the injection rate will be set to zero.

\item \texttt{pump\_elevation}---is the elevation of the multi-aquifer well pump (PUMP\_ELEVATION).  PUMP\_ELEVATION should not be less than the bottom elevation (BOTTOM) of the multi-aquifer well.

\item \texttt{scaling\_length}---height above the pump elevation (SCALING\_LENGTH).  If the simulated well head is below this elevation (pump elevation plus the scaling length), then the pumping rate is reduced.

\item \texttt{AUXILIARY}---keyword for specifying auxiliary variable.

\item \texttt{auxname}---name for the auxiliary variable to be assigned AUXVAL.  AUXNAME must match one of the auxiliary variable names defined in the OPTIONS block. If AUXNAME does not match one of the auxiliary variable names defined in the OPTIONS block the data are ignored.

\item \textcolor{blue}{\texttt{auxval}---value for the auxiliary variable. If the Options block includes a TIMESERIESFILE entry (see the ``Time-Variable Input'' section), values can be obtained from a time series by entering the time-series name in place of a numeric value.}

\end{description}

