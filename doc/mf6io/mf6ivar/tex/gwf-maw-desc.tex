% DO NOT MODIFY THIS FILE DIRECTLY.  IT IS CREATED BY mf6ivar.py 

\item \texttt{auxiliary}---defines an array of one or more auxiliary variable names.  There is no limit on the number of auxiliary variables that can be provided on this line; however, lists of information provided in subsequent blocks must have a column of data for each auxiliary variable name defined here.   The number of auxiliary variables detected on this line determines the value for naux.  Comments cannot be provided anywhere on this line as they will be interpreted as auxiliary variable names.  Auxiliary variables may not be used by the package, but they will be available for use by other parts of the program.  The program will terminate with an error if auxiliary variables are specified on more than one line in the options block.

\item \texttt{BOUNDNAMES}---keyword to indicate that boundary names may be provided with the list of multi-aquifer well cells.

\item \texttt{PRINT\_INPUT}---keyword to indicate that the list of multi-aquifer well information will be written to the listing file immediately after it is read.

\item \texttt{PRINT\_HEAD}---keyword to indicate that the list of multi-aquifer well heads will be printed to the listing file for every stress period in which ``HEAD PRINT'' is specified in Output Control.  If there is no Output Control option and \texttt{PRINT\_HEAD} is specified, then heads are printed for the last time step of each stress period.

\item \texttt{PRINT\_FLOWS}---keyword to indicate that the list of multi-aquifer well flow rates will be printed to the listing file for every stress period time step in which ``BUDGET PRINT'' is specified in Output Control.  If there is no Output Control option and \texttt{PRINT\_FLOWS} is specified, then flow rates are printed for the last time step of each stress period.

\item \texttt{SAVE\_FLOWS}---keyword to indicate that multi-aquifer well flow terms will be written to the file specified with ``BUDGET FILEOUT'' in Output Control.

\item \texttt{HEAD}---keyword to specify that record corresponds to head.

\item \texttt{headfile}---name of the binary output file to write stage information.

\item \texttt{BUDGET}---keyword to specify that record corresponds to the budget.

\item \texttt{FILEOUT}---keyword to specify that an output filename is expected next.

\item \texttt{budgetfile}---name of the binary output file to write budget information.

\item \texttt{NO\_WELL\_STORAGE}---keyword that deactivates inclusion of well storage contributions to the multi-aquifer well package continuity equation.

\item \texttt{FLOWING\_WELLS}---keyword that activates the flowing wells option for the multi-aquifer well package.

\item \texttt{shutdown\_theta}---value that defines the weight applied to discharge rate for wells that limit the water level in a discharging well (defined using the \texttt{HEAD\_LIMIT} keyword in the stress period data). \texttt{SHUTDOWN\_THETA} is used to control discharge rate oscillations when the flow rate from the aquifer is less than the specified flow rate from the aquifer to the well. Values range between 0.0 and 1.0, and larger values increase the weight (decrease under-relaxation) applied to the well discharge rate. The \texttt{head\_limit} option has been included to facilitate backward compatibility with previous versions of MODFLOW but use of the \texttt{rate\_scaling} option instead of the \texttt{head\_limit} option is recommended. By default, \texttt{SHUTDOWN\_THETA} is 0.7.

\item \texttt{shutdown\_kappa}---value that defines the weight applied to discharge rate for wells that limit the water level in a discharging well (defined using the \texttt{HEAD\_LIMIT} keyword in the stress period data). \texttt{SHUTDOWN\_KAPPA} is used to control discharge rate oscillations when the flow rate from the aquifer is less than the specified flow rate from the aquifer to the well. Values range between 0.0 and 1.0, and larger values increase the weight applied to the well discharge rate. The \texttt{head\_limit} option has been included to facilitate backward compatibility with previous versions of MODFLOW but use of the \texttt{rate\_scaling} option instead of the \texttt{head\_limit} option is recommended. By default, \texttt{SHUTDOWN\_KAPPA} is 0.0001.

\item \texttt{TS6}---keyword to specify that record corresponds to a time-series file.

\item \texttt{FILEIN}---keyword to specify that an input filename is expected next.

\item \texttt{ts6\_filename}---defines a time-series file defining time series that can be used to assign time-varying values. See the ``Time-Variable Input'' section for instructions on using the time-series capability.

\item \texttt{OBS6}---keyword to specify that record corresponds to an observations file.

\item \texttt{obs6\_filename}---name of input file to define observations for the MAW package. See the ``Observation utility'' section for instructions for preparing observation input files. Table \ref{table:obstype} lists observation type(s) supported by the MAW package.

\item \texttt{MOVER}---keyword to indicate that this instance of the MAW Package can be used with the Water Mover (MVR) Package.  When the \texttt{MOVER} option is specified, additional memory is allocated within the package to store the available, provided, and received water.

\item \texttt{nmawwells}---integer value specifying the number of multi-aquifer wells that will be simulated for all stress periods.

\item \texttt{wellno}---integer value that defines the well number for this multi-aquifer well entry. \texttt{wellno} must be greater than zero and less than or equal to \texttt{nmawwells}.

\item \texttt{radius}---radius for the multi-aquifer well.

\item \texttt{bottom}---bottom elevation of the multi-aquifer well.

\item \texttt{strt}---starting head for the multi-aquifer well.

\item \texttt{condeqn}---character string that defines the conductance equation that is used to calculate the saturated conductance for the multi-aquifer well. Possible multi-aquifer well \texttt{condeqn} strings include: \texttt{SPECIFIED}--character keyword to indicate the multi-aquifer well saturated conductance will be specified.  \texttt{THEIM}--character keyword to indicate the multi-aquifer well saturated conductance will be calculated using the Theim equation.  \texttt{SKIN}--character keyword to indicate that the multi-aquifer well saturated conductance will be calculated using the screen top and bottom, screen hydraulic conductivity, and skin radius.    \texttt{CUMULATIVE}--character keyword to indicate that the multi-aquifer well saturated conductance will be calculated using a combination of the Theim equation and the screen top and bottom, screen hydraulic conductivity, and skin radius.  \texttt{MEAN}--character keyword to indicate the multi-aquifer well saturated conductance will be calculated using using the aquifer and screen top and bottom, aquifer and screen hydraulic conductivity, and well and skin radius.

\item \texttt{ngwfnodes}---integer value that defines the number of \texttt{GWF} nodes connected to this (\texttt{wellno}) multi-aquifer well. One or more screened intervals can be connected to the same \texttt{GWF} node. \texttt{ngwfnodes} must be greater than zero.

\item \textcolor{blue}{\texttt{aux}---represents the values of the auxiliary variables for each multi-aquifer well. The values of auxiliary variables must be present for each multi-aquifer well. The values must be specified in the order of the auxiliary variables specified in the OPTIONS block.  If the package supports time series and the Options block includes a TIMESERIESFILE entry (see the ``Time-Variable Input'' section), values can be obtained from a time series by entering the time-series name in place of a numeric value.}

\item \texttt{boundname}---name of the multi-aquifer well cell.  \texttt{boundname} is an ASCII character variable that can contain as many as 40 characters.  If \texttt{boundname} contains spaces in it, then the entire name must be enclosed within single quotes.

\item \texttt{wellno}---integer value that defines the well number for this multi-aquifer well entry. \texttt{wellno} must be greater than zero and less than or equal to \texttt{nmawwells}.

\item \texttt{icon}---integer value that defines the \texttt{GWF} connection number for this multi-aquifer well connection entry. \texttt{iconn} must be greater than zero and less than or equal to \texttt{ngwfnodes} for multi-aquifer well \texttt{wellno}.

\item \texttt{cellid}---is the cell identifier, and depends on the type of grid that is used for the simulation.  For a structured grid that uses the DIS input file, \texttt{cellid} is the layer, row, and column.   For a grid that uses the DISV input file, \texttt{cellid} is the layer and cell2d number.  If the model uses the unstructured discretization (DISU) input file, then \texttt{cellid} is the node number for the cell.

\item \texttt{scrn\_top}---value that defines the top elevation of the screen for the multi-aquifer well connection.  \texttt{scrn\_top} can be any value if \texttt{condeqn} is \texttt{SPECIFIED} or \texttt{THEIM}. If the specified \texttt{scrn\_top} is greater than the top of the \texttt{GWF} cell it is set equal to the top of the cell.

\item \texttt{scrn\_bot}---value that defines the bottom elevation of the screen for the multi-aquifer well connection.  \texttt{scrn\_bot} can be any value if \texttt{condeqn} is \texttt{SPECIFIED} or \texttt{THEIM}. If the specified \texttt{scrn\_bot} is less than the bottom of the \texttt{GWF} cell it is set equal to the bottom of the cell.

\item \texttt{hk\_skin}---value that defines the skin (filter pack) hydraulic conductivity (if \texttt{condeqn} for the multi-aquifer well is \texttt{SKIN}, \texttt{CUMULATIVE}, or \texttt{MEAN}) or conductance (if \texttt{condeqn} for the multi-aquifer well is \texttt{SPECIFIED}) for each \texttt{GWF} node connected to the multi-aquifer well (\texttt{ngwfnodes}). \texttt{hk\_skin} can be any value if \texttt{condeqn} is \texttt{THEIM}.

\item \texttt{radius\_skin}---real value that defines the skin radius (filter pack radius) for the multi-aquifer well. \texttt{radius\_skin} can be any value if \texttt{condeqn} is \texttt{SPECIFIED} or \texttt{THEIM}. Otherwise, \texttt{radius\_skin} must be greater than \texttt{radius} for the multi-aquifer well.

\item \texttt{iper}---integer value specifying the starting stress period number for which the data specified in the PERIOD block apply.  \texttt{iper} must be less than \texttt{nper} in the TDIS Package and greater than zero.  The \texttt{iper} value assigned to a stress period block must be greater than the \texttt{iper} value assigned for the previous block.

\item \texttt{wellno}---integer value that defines the well number for this multi-aquifer well entry. \texttt{wellno} must be greater than zero and less than or equal to \texttt{nmawwells}.

\item \texttt{mawsetting}---line of information that is parsed into a keyword and values.  Keyword values that can be used to start the \texttt{mawsetting} string include: \texttt{STATUS}, \texttt{FLOWING\_WELL}, \texttt{RATE}, \texttt{WELL\_HEAD}, \texttt{HEAD\_LIMIT}, \texttt{SHUT\_OFF}, \texttt{RATE\_SCALING}, and \texttt{AUXILIARY}.

\begin{lstlisting}[style=blockdefinition]
STATUS <status>
FLOWING_WELL <fwelev> <fwcond> 
RATE <@rate@>
WELL_HEAD <@well_head@>
HEAD_LIMIT <head_limit>
SHUT_OFF <minrate> <maxrate> 
RATE_SCALING <pump_elevation> <scaling_length> 
AUXILIARY <auxname> <@auxval@> 
\end{lstlisting}

\item \texttt{status}---keyword option to define well status.  \texttt{status} can be \texttt{ACTIVE}, \texttt{INACTIVE}, or \texttt{CONSTANT}. By default, \texttt{status} is \texttt{ACTIVE}.

\item \texttt{FLOWING\_WELL}---keyword to indicate the well is a flowing well.  The \texttt{flowing\_well} option can be used to simulate flowing wells when the simulated well head exceeds the specified drainage elevation.

\item \texttt{fwelev}---elevation used to determine whether or not the well is flowing.

\item \texttt{fwcond}---conductance used to calculate the discharge of a free flowing well.  Flow occurs when the head in the well is above the well top elevation (\texttt{fwelev}).

\item \textcolor{blue}{\texttt{rate}---is the volumetric pumping rate for the multi-aquifer well. A positive value indicates recharge and a negative value indicates discharge (pumping). \texttt{rate} only applies to active (\texttt{IBOUND}$>0$) multi-aquifer wells. If the Options block includes a \texttt{TIMESERIESFILE} entry (see the ``Time-Variable Input'' section), values can be obtained from a time series by entering the time-series name in place of a numeric value. By default, the \texttt{rate} for each multi-aquifer well is zero.}

\item \textcolor{blue}{\texttt{well\_head}---is the head in the multi-aquifer well. \texttt{well\_head} is only applied to constant head (\texttt{STATUS} is \texttt{CONSTANT}) and inactive (\texttt{STATUS} is \texttt{INACTIVE}) multi-aquifer wells. If the Options block includes a \texttt{TIMESERIESFILE} entry (see the ``Time-Variable Input'' section), values can be obtained from a time series by entering the time-series name in place of a numeric value.}

\item \texttt{head\_limit}---is the limiting water level (head) in the well, which is the minimum of the well \texttt{rate} or the well inflow rate from the aquifer. \texttt{head\_limit} is only applied to discharging wells (\texttt{rate}$<0$). \texttt{head\_limit} can be deactivated by specifying the text string `\texttt{off}'. The \texttt{head\_limit} option is based on the \texttt{head\_limit} functionality available in the MNW2~\citep{konikow2009} package for MODFLOW-2005. The \texttt{head\_limit} option has been included to facilitate backward compatibility with previous versions of MODFLOW but use of the \texttt{rate\_scaling} option instead of the \texttt{head\_limit} option is recommended. By default, \texttt{head\_limit} is `\texttt{off}'.

\item \texttt{SHUT\_OFF}---keyword for activating well shut off capability.  Subsequent values define the minimum and maximum pumping rate that a well must exceed to shutoff or reactivate a well, respectively, during a stress period. \texttt{shut\_off} is only applied to discharging wells (\texttt{rate}$<0$) and if \texttt{head\_limit} is specified (not set to `\texttt{off}').  If \texttt{head\_limit} is specified, \texttt{shut\_off} can be deactivated by specifying a minimum value equal to zero. The \texttt{shut\_off} option is based on the \texttt{shut\_off} functionality available in the MNW2~\citep{konikow2009} package for MODFLOW-2005. The \texttt{shut\_off} option has been included to facilitate backward compatibility with previous versions of MODFLOW but use of the \texttt{rate\_scaling} option instead of the \texttt{shut\_off} option is recommended. By default, \texttt{shut\_off} is not used.

\item \texttt{minrate}---is the minimum rate that a well must exceed to shutoff a well during a stress period. The well will shut down during a time step if the flow rate to the well from the aquifer is less than \texttt{minrate}. If a well is shut down during a time step, reactivation of the well cannot occur until the next time step to reduce oscillations. \texttt{minrate} must be less than \texttt{maxrate}.

\item \texttt{maxrate}---is the maximum rate that a well must exceed to reactivate a well during a stress period. The well will reactivate during a timestep if the well was shutdown during the previous time step and the flow rate to the well from the aquifer exceeds \texttt{maxrate}. Reactivation of the well cannot occur until the next time step if a well is shutdown to reduce oscillations. \texttt{maxrate} must be greater than \texttt{minrate}.

\item \texttt{RATE\_SCALING}---activate rate scaling.  If \texttt{rate\_scaling} is specified, both \texttt{pump\_elevation} and \texttt{scaling\_length} must be specified. \texttt{rate\_scaling} cannot be used with \texttt{head\_limit}.

\item \texttt{pump\_elevation}---is the elevation of the multi-aquifer well pump (\texttt{pump\_elevation}).  \texttt{pump\_elevation} cannot be less than the bottom elevation (\texttt{bottom}) of the multi-aquifer well. By default, \texttt{pump\_elevation} is set equal to the bottom of the largest \texttt{GWF} node number connected to a MAW well.

\item \texttt{scaling\_length}---height above the pump elevation (\texttt{scaling\_length}) below which the pumping rate is reduced.  The default value for \texttt{scaling\_length} is the well radius.

\item \texttt{AUXILIARY}---keyword for specifying auxiliary variable.

\item \texttt{auxname}---name for the auxiliary variable to be assigned \texttt{auxval}.  \texttt{auxname} must match one of the auxiliary variable names defined in the \texttt{OPTIONS} block. If \texttt{auxname} does not match one of the auxiliary variable names defined in the \texttt{OPTIONS} block the data are ignored.

\item \textcolor{blue}{\texttt{auxval}---value for the auxiliary variable. If the Options block includes a \texttt{TIMESERIESFILE} entry (see the ``Time-Variable Input'' section), values can be obtained from a time series by entering the time-series name in place of a numeric value.}


