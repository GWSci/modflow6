% DO NOT MODIFY THIS FILE DIRECTLY.  IT IS CREATED BY mf6ivar.py 

\item \textbf{Block: OPTIONS}

\begin{description}
\item \texttt{PRINT\_INPUT}---keyword to indicate that the list of VKD cells and information will be written to the listing file immediately after it is read.

\end{description}
\item \textbf{Block: DIMENSIONS}

\begin{description}
\item \texttt{numvkd}---integer value specifying the number of VKD cells. There must be NUMVKD entries in the PACKAGEDATA block.

\item \texttt{numelevs}---integer value specifying the number of VKD inflections. There must be NUMELEVS entries for both EIP and KIP for every cell defined in PACKAGEDATA block. NB 'implicit inflection points' will be placed at the top and bottom elevations of each cell defined in PACKAGEDATA block.

\end{description}
\item \textbf{Block: PACKAGEDATA}

\begin{description}
\item \texttt{cellid}---is the cell identifier, and depends on the type of grid that is used for the simulation.  For a structured grid that uses the DIS input file, CELLID is the layer, row, and column.   For a grid that uses the DISV input file, CELLID is the layer and CELL2D number.  If the model uses the unstructured discretization (DISU) input file, CELLID is the node number for the cell.

\item \texttt{eip}---elevation of VKD inflection point.  This number is an elevation (relative to the same datum used for cell tops and bottoms) in model length units.

\item \texttt{kip}---hydraulic conductivity factor at VKD inflection point.

\end{description}

