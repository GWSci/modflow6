% DO NOT MODIFY THIS FILE DIRECTLY.  IT IS CREATED BY mf6ivar.py 

\item \textbf{Block: OPTIONS}

\begin{description}
\item \texttt{SAVE\_FLOWS}---keyword to indicate that cell-by-cell flow terms will be written to the file specified with ``BUDGET SAVE FILE'' in Output Control.

\item \texttt{ndelaycells}---number of nodes used to discretize the the delay intebed thickness to approximate the head distributions in systems of delay interbeds. If not specified, then a default value of 19 is assigned.

\item \texttt{INTERBED\_STRESS\_OFFSET}---keyword to indicate that an initial stress offset specified in the \texttt{GRIDDATA} block will be applied to each interbed.

\item \texttt{GEO\_STRESS\_OFFSET}---keyword to indicate that a geostatic stress offset will be specified in the \texttt{PERIOD} block.

\item \texttt{STORAGECOEFFICIENT}---keyword to indicate that the elastic skeletal specific storage (\texttt{sse}) and inelastic skeletal specific storage (\texttt{ssv}) coefficients are specified rather than the recompression (\texttt{CR}) and compresion (\texttt{CC}) indices.

\item \texttt{CONSTANT\_THICKNESS}---keyword to indicate that the thickness and void ratio of the interbed will not be varied during the simulation.

\item \texttt{CELL\_FRACTION}---keyword to indicate that the thickness of interbeds will be specified in terms of the fraction of cell thickness.

\item \texttt{COMPACTION}---keyword to specify that record corresponds to the compaction.

\item \texttt{FILEOUT}---keyword to specify that an output filename is expected next.

\item \texttt{compactionfile}---name of the binary output file to write compaction information.

\item \texttt{TS6}---keyword to specify that record corresponds to a time-series file.

\item \texttt{FILEIN}---keyword to specify that an input filename is expected next.

\item \texttt{ts6\_filename}---defines a time-series file defining time series that can be used to assign time-varying values. See the ``Time-Variable Input'' section for instructions on using the time-series capability.

\item \texttt{OBS6}---keyword to specify that record corresponds to an observations file.

\item \texttt{obs6\_filename}---name of input file to define observations for the IBC package. See the ``Observation utility'' section for instructions for preparing observation input files. Table \ref{table:obstype} lists observation type(s) supported by the IBC package.

\end{description}
\item \textbf{Block: DIMENSIONS}

\begin{description}
\item \texttt{nibccells}---is the number of IBC cells.  More than 1 IBC cell can be assigned to a GWF cell; however, only 1 GWF cell can be assigned to a single IBC cell.

\end{description}
\item \textbf{Block: GRIDDATA}

\begin{description}
\item \texttt{sgm}---is specific gravity of moist or unsaturated sediments.

\item \texttt{sgs}---is specific gravity of saturated sediments.

\end{description}
\item \textbf{Block: PACKAGEDATA}

\begin{description}
\item \texttt{ibcno}---integer value that defines the IBC interbed number associated with the specified PACKAGEDATA data on the line. \texttt{isubno} must be greater than zero and less than or equal to \texttt{nsubcells}.  IBC information must be specified for every IBC cell or the program will terminate with an error.  The program will also terminate with an error if information for a IBC intebed number is specified more than once.

\item \texttt{cellid}---is the cell identifier, and depends on the type of grid that is used for the simulation.  For a structured grid that uses the DIS input file, \texttt{cellid} is the layer, row, and column.   For a grid that uses the DISV input file, \texttt{cellid} is the layer and cell2d number.  If the model uses the unstructured discretization (DISU) input file, then \texttt{cellid} is the node number for the cell.

\item \texttt{cdelay}---character string that defines the subsidence delay type for the IBC cell. Possible subsidence package \texttt{cdelay} strings include: \texttt{NODELAY}--character keyword to indicate that delay will not be simulated in the IBC cell.  \texttt{DELAY}--character keyword to indicate that delay will be simulated in the IBC cell.

\item \texttt{pcs0}---is the initial preconsolidation stress or initial offset from the calculated initial effective stress in the IBC interbed, in units of height of a column of water. \texttt{pcs0} is the initial offset from the calculated initial effective stress if \texttt{INTERBED\_STRESS\_OFFSET} is specified in the \texttt{OPTIONS} block.

\item \texttt{thick\_frac}---is the interbed thickness or cell fraction of the IBC interbed. Interbed thickness is specified as a fraction of the cell thickness if \texttt{cell\_fraction} is specified in the \texttt{OPTIONS} block.

\item \texttt{rnb}---is the interbed material factor $rn_{equiv}$ for the system of delay interbeds for interbed \texttt{ibcno}. \texttt{rnb} must be greater than or equal to 1 if \texttt{cdelay} is \texttt{delay}. Otherwise, \texttt{rnb} can be any value.

\item \texttt{ssv\_cc}---is the initial inelastic skeletal specific storage or compression index in the IBC interbed. The initial inelastic skeletal specific storage is specified if \texttt{storage\_coefficient} is specified in the \texttt{OPTIONS} block.

\item \texttt{sse\_cr}---is the initial elastic skeletal specific storage or recompression index in the IBC interbed. The initial elastic skeletal specific storage is specified if \texttt{storage\_coefficient} is specified in the \texttt{OPTIONS} block.

\item \texttt{void}---is the initial initial void ratio of the IBC interbed.

\item \texttt{kv}---is the vertical hydraulic conductivity of the IBC interbed. \texttt{kv} must be greater than 0 if \texttt{cdelay} is \texttt{delay}. Otherwise, \texttt{kv} can be any value.

\item \texttt{h0}---is the initial head in the IBC interbed. \texttt{h0} can be anyvalue if \texttt{cdelay} is \texttt{nodelay}.

\item \texttt{boundname}---name of the IBC cell cell.  \texttt{boundname} is an ASCII character variable that can contain as many as 40 characters.  If \texttt{boundname} contains spaces in it, then the entire name must be enclosed within single quotes.

\end{description}
\item \textbf{Block: PERIOD}

\begin{description}
\item \texttt{iper}---integer value specifying the starting stress period number for which the data specified in the PERIOD block apply.  \texttt{iper} must be less than or equal to \texttt{nper} in the TDIS Package and greater than zero.  The \texttt{iper} value assigned to a stress period block must be greater than the \texttt{iper} value assigned for the previous PERIOD block.  The information specified in the PERIOD block will continue to apply for all subsequent stress periods, unless the program encounters another PERIOD block.

\item \texttt{sig0}---is the stress offset for the cell. \texttt{sig0} is added to the calculated geostatic stress for the cell. \texttt{glo} is specified only if \texttt{geo\_stress\_offset} is specified in the \texttt{OPTIONS} block.

\end{description}

