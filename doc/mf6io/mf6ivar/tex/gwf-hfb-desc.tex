% DO NOT MODIFY THIS FILE DIRECTLY.  IT IS CREATED BY mf6ivar.py 

\item \textbf{Block: OPTIONS}

\begin{description}
\item \texttt{PRINT\_INPUT}---keyword to indicate that the list of horizontal flow barriers will be written to the listing file immediately after it is read.

\end{description}
\item \textbf{Block: DIMENSIONS}

\begin{description}
\item \texttt{maxhfb}---integer value specifying the maximum number of horizontal flow barriers that will be entered in this input file.  The value of MAXHFB is used to allocate memory for the horizontal flow barriers.

\end{description}
\item \textbf{Block: PERIOD}

\begin{description}
\item \texttt{iper}---integer value specifying the starting stress period number for which the data specified in the PERIOD block apply.  IPER must be less than or equal to NPER in the TDIS Package and greater than zero.  The IPER value assigned to a stress period block must be greater than the IPER value assigned for the previous PERIOD block.  The information specified in the PERIOD block will continue to apply for all subsequent stress periods, unless the program encounters another PERIOD block.

\item \texttt{cellid1}---identifier for the first cell.  For a structured grid that uses the DIS input file, CELLID1 is the layer, row, and column numbers of the cell.   For a grid that uses the DISV input file, CELLID1 is the layer number and CELL2D number for the two cells.  If the model uses the unstructured discretization (DISU) input file, then CELLID1 is the node numbers for the cell.  The barrier is located between cells designated as CELLID1 and CELLID2.  For models that use the DIS and DISV grid types, the layer number for CELLID1 and CELLID2 must be the same.  For all grid types, cells must be horizontally adjacent or the program will terminate with an error.

\item \texttt{cellid2}---identifier for the second cell. See CELLID1 for description of how to specify.

\item \texttt{hydchr}---is the hydraulic characteristic of the horizontal-flow barrier. The hydraulic characteristic is the barrier hydraulic conductivity divided by the width of the horizontal-flow barrier. If hydraulic characteristic is negative, then it acts as a multiplier to the conductance between the two model cells specified as containing a barrier. For example, if the value for HYDCHR was specified as 1.5, the conductance calculated for the two cells would be multiplied by 1.5.

\end{description}

